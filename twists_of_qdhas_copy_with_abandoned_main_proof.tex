\documentclass[10pt]{article}

\usepackage[a4paper, margin=2.25cm, top=3cm ]{geometry}
%\usepackage[margin=1in]{geometry}

\usepackage[all]{xy}
\usepackage[nottoc]{tocbibind}
\usepackage{amsmath,amsthm,amsfonts,longtable,verbatim,tikz-cd,multicol,amssymb,wasysym,setspace,graphicx,titlesec,imakeidx,dynkin-diagrams,etoolbox}
% Packages incompatible with plastex.
%\usepackage{xspace}
%\usepackage{mathtools}
%\usepackage{mathrsfs}
\usepackage[inline]{enumitem}
\usepackage[colorlinks,unicode]{hyperref}
\urlstyle{same}


\setcounter{tocdepth}{2}
%\titleformat{\subsubsection}[runin]{}{}{}{}[]
\titleformat{\subsubsection}[runin]{\normalfont\bfseries}{\thesubsubsection.}{5pt}{}{}
\titlespacing{\subsubsection}{0pt}{5pt}{10pt}

\onehalfspacing

\newcommand{\bb}{\medbreak}
\newcommand{\nt}{\noindent}
\newcommand{\R}{\mathbb{R}}
\newcommand{\Q}{\mathbb{Q}}
\newcommand{\Z}{\mathbb{Z}}
\newcommand{\N}{\mathbb{N}}
\newcommand{\Cc }{\mathbb{C}}
\newcommand{\rt}{\xrightarrow{}}
\newcommand{\xrt}{\xrightarrow}
\newcommand{\cg}{\mathfrak{g}}
\newcommand{\cd}{\cdot}
\newcommand{\id}{\text{id}}
\newcommand{\gl}{\mathfrak{gl}}
\newcommand{\GL}{\text{GL}}
\newcommand{\ad}{\text{ad}}
\newcommand{\End}{\text{End}}
\newcommand{\Der}{\text{Der}}
\newcommand{\chr}{\text{char}}
\newcommand{\tr}{\text{tr}}
\newcommand{\rad}{\text{Rad}}
\newcommand{\prim}{\text{prim}}
\newcommand{\im}{\text{Im}}
\newcommand{\spn}{\text{span}}
\newcommand{\gsl}{\mathfrak{sl}}
\newcommand{\iso}{\text{Iso}}
\newcommand{\ind}{\text{Ind}}
\newcommand{\ob}{\text{ob}}
\newcommand{\DD}{\mathcal{D}}
\newcommand{\aut}{\text{Aut}}
\newcommand{\rank}{\text{rank}}
\newcommand{\ux}{\underline{x}}
\newcommand{\uy}{\underline{y}}
\newcommand{\Rep}{\text{Rep}}
\newcommand{\Tr}{\text{Tr}}
\newcommand{\Mod}{\text{-Mod}}
\newcommand{\bbw}{\overline{\overline{W}}}
\newcommand{\bw}{\overline{W}}
\newcommand{\bq}{\overline{q}}
\newcommand{\op}{\text{op}}
\newcommand{\Alg}{\text{Alg}}
\newcommand{\gr}{\text{gr}}
\newcommand{\hh}{\mathfrak{h}}
\newcommand{\HH}{\text{HH}}
\newcommand{\al}{\alpha}
\newcommand{\dash}{\text{-}}
\newcommand{\CL}{\text{Cl}}
\newcommand{\so}{\mathfrak{so}}
\newcommand{\dih}{\text{Dih}}
\newcommand{\dic}{\text{Dic}}
\newcommand{\spin}{\text{Spin}}
\newcommand{\pin}{\text{Pin}}
\newcommand{\SO}{\text{SO}}
\newcommand{\UU}{\text{U}}
\newcommand{\OO}{\text{O}}
\newcommand{\I}{\mathbb{I}}
\newcommand{\E}{\mathcal{E}}
\newcommand{\F}{\mathcal{F}}


% for drawing dynkin diagrams
\def\row#1/#2!{#1_{\IfStrEq{#2}{}{n}{#2}} & \dynkin{#1}{#2}\\}
\newcommand{\tble}[1]{
   \renewcommand*\do[1]{\row##1!}
   \[
      \begin{array}{ll}\docsvlist{#1}\end{array}
   \]
}

% this is to highlight words that are being defined and enter.
\newcommand{\define}[1]{\textbf{#1}\index{#1}}
\newcommand{\definex}[1]{\textbf{#1}}

% The next two lines define a reasonable looking not divides sign.
\DeclareMathSymbol{\nmid}{\mathrel}{AMSb}{"2D}
\newcommand{\notdiv}{\nmid}

% Make the tilde command wider
\renewcommand{\tilde}{\widetilde}

% the following two commands change the way the footnote symbol is
% made to be old fashioned: *, dagger, etc.  
\renewcommand{\thefootnote}{\fnsymbol{footnote}}

% We start with things like lemmas, theorems, etc.
\newtheorem{lemma}{Lemma}[section]
\newtheorem{theorem}[lemma]{Theorem}
\newtheorem{cor}[lemma]{Corollary}
\newtheorem*{scholium}{Scholium}
\newtheorem{proposition}[lemma]{Proposition}

% Now we create things like definitions, examples, comments, etc.  
\theoremstyle{definition}
\newtheorem{definition}[lemma]{Definition}
\newtheorem{example}[lemma]{Example}
\newenvironment{comments}{}{}
\newenvironment{solution}{\smallskip\par\noindent\emph{Solution: }}{}
\newtheorem{note}[lemma]{Note}





\makeindex[intoc]


\title{Twists of quantum Drinfeld Hecke algebras}
\author{Edward Jones-Healey}
\date{}

\counterwithin*{section}{part}



\begin{document}

\maketitle

\begin{abstract} In the preprint \cite{twistsrcas} it was shown that certain rational and braided Cherednik algebras are related via a Drinfeld twist. We compare this to results from Naidu and Witherspoon \cite{2011arXiv11115243N} which show quantum Drinfeld Hecke algebras, which generalise the rational and braided Cherednik algebras, arise as certain deformations of their associated graded algebras. In particular we show that if a pair of quantum Drinfeld Hecke algebras are related by a Drinfeld twist then their associated graded algebras, which are isomorphic to skew group algebras, are related by the same twist. Conversely, if two skew group algebras are related by a Drinfeld twist, then it is possible to preserve the twisting result when the algebras are deformed by certain ``equivariant'' Hochschild $2$-cocycles into quantum Drinfeld Hecke algebras.
\end{abstract}

\tableofcontents

%\section{Summary}
In Part \ref{part_3} I have tried to elaborate further on the main twisting result of the preprint \cite{twistsrcas} being written by Bazlov, Berenstein, McGaw and myself. In that preprint we showed two particular instances of braided Cherednik algebras are related by a ``Drinfeld twist''. Here I make use of a paper by Naidu and Witherspoon \cite{2011arXiv11115243N} in which the class of ``quantum Drinfeld Hecke algebras'' (which include braided Cherednik algebras) arise as deformations of their associated graded rings. In Section \ref{claim_3_sec} we prove that, under certain conditions, Drinfeld twists and deformations are compatible, therefore allowing us to derive twisting results between quantum Drinfeld Hecke algebras from simpler twisting results between graded rings.


%\include{part_3}

\section{Introduction}
\subsection{Preliminaries}
In this section we review several definitions and results from Naidu and Witherspoon \cite{2011arXiv11115243N} and the preprint \cite{twistsrcas}, before outlining our new results in Section \ref{new_results_sec}. Proofs will be the subject of the remainder of the article. Unless otherwise stated assume all objects in this text are $\Cc $-linear, with tensor products also taken over $\Cc $.\bb

\nt Given a $\Cc $-vector space $V$ with basis $v_1,\dots,v_n$ and an $n\times n$-matrix $q=(q_{ij})$ such that $q_{ii}=1,q_{ij}q_{ji}=1$, we define a \define{skew symmetric algebra} as $S_q(V):=\Cc \langle v_1,\dots,v_n|\ v_iv_j=q_{ij}v_jv_i\ \forall i,j \rangle$, and a \define{skew exterior algebra} as $\bigwedge_q(V):=\Cc  \langle v_1,\dots,v_n|\ v_i v_j=-q_{ij} v_j v_i\rangle$. Now let $G$ be a group that is equipped with an action on $V$ that extends to an action by algebra automorphisms on $S_q(V)$.

\begin{definition}\label{qDHA_defn} Let $\kappa:V\times V\rt \Cc  G$ be a bilinear map such that $\kappa(v_i,v_j)=-q_{ij}\kappa(v_j,v_i)$, and define
$$H_{q,\kappa}:=T(V)\rtimes G/\langle v_i v_j-q_{ij}v_j v_i-\kappa(v_i,v_j)\rangle$$
Recall $T(V)\rtimes G$ has underlying vector space $T(V)\otimes \Cc  G$, identity $1_\Cc  \otimes 1_G$, and product: $(a\otimes g)\cdot (a'\otimes g')=(a\cdot (g\rhd a'))\otimes gg'$ for $a,a'\in T(V),\ g,g'\in G$, and the action of $G$ on $T(V)$ is given by naturally extending the action of $G$ on $V$. With $v_i$ in degree $1$ and $g\in G$ in degree $0$, $H_{q,\kappa}$ is a filtered algebra. $H_{q,\kappa}$ is called a \define{quantum Drinfeld Hecke algebra} if its associated graded algebra is isomorphic to the skew group algebra $S_q(V)\rtimes G$. Levandovsky and Shepler \cite{2011arXiv11114975L} give the conditions on $\kappa$ that ensure $H_{q,\kappa}$ is a quantum Drinfeld Hecke algebra.
\end{definition}

\nt Drinfeld Hecke algebras (also known as graded Hecke algebras) arise in the special case $q_{ij}=1\ \forall i,j$. These include the symplectic reflection algebras, and hence all rational Cherednik algebras too. Of import to us is that the more general \textit{quantum} Drinfeld Hecke algebras include the \textit{braided} Cherednik algebras, a class of algebras introduced by Bazlov and Berenstein in \cite{2008arXiv0806.0867B}. One example of such an algebra is the ``negative braided Cherednik algebra", which is, in a sense, the anticommutative analogue of a rational Cherednik algebra. An explicit characterisation of the negative braided Cherednik algebra as a quantum Drinfeld Hecke algebra was given in \cite{2011arXiv11115243N} (Remark 7.4).\bb

\nt In Naidu and Witherspoon \cite{2011arXiv11115243N} the quantum Drinfeld Hecke algebras were characterised as deformations of the skew group algebras of the form $S_q(V)\rtimes G$. For a $\Cc $-algebra $R$, a \define{deformation over $\Cc [t]$} is an associative $\Cc [t]$-algebra with underlying vector space $\Cc [t]\otimes_\Cc  R$ and product:
$$r\ast s=rs +\mu_1(r\otimes s)t+\mu_2(r\otimes s)t^2+\dots$$
for $\Cc $-linear maps $\mu_i:R\otimes R\rt R,\ i\in \N$, extended linearly over $\Cc [t]$. It turns out that associativity of $\ast$ implies $\mu_1$ is a Hochschild $2$-cocycle (see Definition \ref{equivariant_cocycle}). A ``quantum Drinfeld Hecke algebra over $\Cc  [t]$" will be $\Cc [t]$-linear analogue which specialises at $t=1$ to the quantum Drinfeld Hecke algebras of Definition \ref{qDHA_defn}.

\begin{theorem}[\cite{2011arXiv11115243N}, Theorem 2.2] The quantum Drinfeld Hecke algebras over $\Cc [t]$ are precisely those deformations over $\Cc [t]$ of $S_q(V)\rtimes G$ where $\deg(\mu_i)=-2i$.
\end{theorem}

\nt Note that $\deg(\mu_i)=-2i$ means that the map $\mu_i:(S_q(V)\rtimes G)^{\otimes 2}\rt S_q(V)\rtimes G$ is a map of graded algebras which sends elements of degree $j$ to elements of degree $j-2i$. By this theorem we can associate a Hochschild $2$-cocycle $\mu_1$ (which is a map of degree $-2$) to each quantum Drinfeld Hecke algebra.
\begin{definition}\label{constant} \cite{2011arXiv11115243N}
A \define{constant Hochschild $2$-cocycle} is a Hochschild $2$-cocycle of a skew group algebra $S_q(V)\rtimes G$ that is degree $-2$ as a map of graded algebras.
\end{definition}

\nt The following result shows that each constant Hochschild $2$-cocycles conversely gives rise to a quantum Drinfeld Hecke algebra:
\begin{theorem}[\cite{2011arXiv11115243N}, Theorem 4.4,4.6]\label{quantum_dh_alg} When the action of $G$ on $V$ extends to an action on $\bigwedge_q(V)$ by algebra automorphisms, each constant Hochschild $2$-cocycle $\alpha$ of $S_q(V)\rtimes G$ lifts to deformation of $S_q(V)\rtimes G$ over $\Cc [t]$ which is isomorphic to a quantum Drinfeld Hecke algebra $Q_\alpha$ with associated graded algebra isomorphic to $S_q(V)\rtimes G$.
\end{theorem}

\nt Finally, let us we recall the main theorem from the new preprint.

\begin{theorem}[Bazlov, Berenstein, Jones-Healey, McGaw \cite{twistsrcas}]\label{our_theorem} For complex reflection group $G=G(m,p,n)$ with $m$ even and reflection representation $V$, the corresponding rational Cherednik algebra $H_{c}(G)$ is a $\Cc  T$-module algebra for $T:=(C_2)^n$. Additionally there is a cocycle $\mu$ of $\Cc  T$ such that the Drinfeld twist of $H_c(G)$ by $\mu$ is isomorphic to the negative braided Cherednik algebra $\underline{H}_{\underline{c}}(\mu(G))$.
\end{theorem}

\subsection{Statements of new results}\label{new_results_sec}

Here we outline our new results as a series of three claims, whose precise statements and proofs will be given in the next sections.\bb

\nt \definex{Claim 1:} if two quantum Drinfeld Hecke algebras are related by Drinfeld twist, then their associated graded algebras are related by the same twist.\bb

\nt Using Theorem \ref{our_theorem} to provide an example of quantum Drinfeld Hecke algebras related by a twist we can deduce from the claim that:
\begin{equation}\label{twist_skew_sym_algs_1}
(S(V\oplus V^*)\rtimes G)_\mu=S_{-1}(V\oplus V^*)\rtimes \mu(G)
\end{equation}
Proving Claim 1 is the subject of Section \ref{claim_1_sec}. Next we consider the opposite direction, taking a pair of skew group algebras related by twist - as in \eqref{twist_skew_sym_algs_1} - and use these to produce a pair of quantum Drinfeld Hecke algebras related by a twist. In particular, suppose we have skew group algebras $A=S_q(V)\rtimes G$ and $B=S_{q'}(V')\rtimes G'$, which are:
\begin{itemize}
  \item related by a Drinfeld twist, i.e. $A$ is an $H$-module algebra for some Hopf algebra $H$ and there exists a counital $2$-cocycle $\mu$ of $H$ such that $A_\mu\cong B$,
  
  \item and such that $G$ and $G'$ act by algebra automorphisms on $\bigwedge_q(V)$ and $\bigwedge_{q'}(V')$ respectively (this allows us to apply Theorem \ref{quantum_dh_alg} in order to lift Hochschild $2$-cocycles to a quantum Drinfeld Hecke algebras).
\end{itemize}

%\nt By Equation \eqref{twist_skew_sym_algs_1} our canonical example of such a pair of algebras is $A=S(V\oplus V^*)\rtimes G$ and $B=S_{-1}(V\oplus V^*)\rtimes \mu(G)$. With $A$ and $B$ given as above, we make several further claims:\bb

\nt \definex{Claim 2:} the cocycle $\mu$ of $H$ induces an isomorphism between the space of $H$-equivariant Hochschild $2$-cocycles of $A$ (see Definition \ref{equivariant_defn}) and the space of $H_\mu$-equivariant Hochschild $2$-cocycles for $B$, i.e. we have a map
\begin{equation}\label{cocycle_map}
\HH_\rhd^2(A)\xrt{\sim} \HH_\rhd^2(B),\ [\alpha]\mapsto [\alpha_\mu]
\end{equation}
Additionally if $\alpha$ is constant (as in Definition \ref{constant}), then $\alpha_\mu$ is constant. We prove this in Section \ref{claim_2_sec}.\bb

\nt \definex{Claim 3:} Each $H$-equivariant Hochschild $2$-cocycle $\alpha$ of $A$ can be used to ``lift'' the twisting result $A_\mu\cong B$ to a twisting result between quantum Drinfeld Hecke algebras. In particular we have $(Q_\alpha)_\mu \cong Q_{\alpha_\mu}$, where $\alpha_\mu$ is given via \eqref{cocycle_map} and $Q_\alpha$, $Q_{\alpha_\mu}$ are quantum Drinfeld Hecke algebras deforming $A$ and $B$ respectively via Theorem \ref{quantum_dh_alg}.\bb

%The Drinfeld twist result $A_\mu\cong B$ ``lifts" via each pair of Hochschild $2$-cocycles $(\alpha, \alpha_\mu)$ arising through Claim 2 to produce a new twisting result between quantum Drinfeld Hecke algebras. In particular the new twisting result is between the quantum Drinfeld Hecke algebras $Q_\alpha, Q_{\alpha_\mu}$ generated (via Theorem \ref{quantum_dh_alg}) by $(A,\alpha)$ and $(B,\alpha_\mu)$ respectively. We address this claim in Section \ref{claim_3_sec}.\bb

\nt  This claim can be illustrated as a commuting (up to isomorphism) diagram of twists and deformations: 
%Inspecting the diagram below, Claim 2 essentially states that if we begin with the bottom horizontal arrow of the diagram, then there is a 1-1 correspondence between the set of vertical arrows we can use on the LHS of the diagram and the set of vertical arrows we can use on the RHS of the diagram. Claim 3 tells us that given the bottom arrow, a choice of left vertical arrow, and the corresponding right vertical arrow given by the 1-1 correspondence of Claim 2, then we can complete the diagram to produce a commuting square\footnote{Beware this is not a commuting diagram in the categorical sense since none of the arrows are algebra homomorphisms.} of deformations/twists using a top horizontal arrow which is just another Drinfeld twist by $\mu$.
$$\xymatrix@C=8em{
  Q_\alpha \ar[r]^{\text{Drinfeld twist by }\mu} & (Q_\alpha)_\mu \cong Q_{\alpha_\mu}\\
  S_q(V)\rtimes G \ar[u]^{\text{deformation by }\alpha} \ar[r]_{\text{Drinfeld twist by }\mu} & S_{q'}(V')\rtimes G' \ar[u]_{\text{deformation by }\alpha_\mu}
}$$
%To summarise, in Claim 1 we are assuming we have only the top horizontal edge in this diagram (i.e. a pair of quantum Drinfeld Hecke algebras related by Drinfeld twist) and deduce the twisting result given by the bottom horizontal edge by considering associated graded algebras. Alternatively, we assume we have the bottom horizontal edge (i.e. a pair of skew group algebras related by Drinfeld twist). Through Claim 2 we find that each choice of left vertical arrow for this diagram (given by an equivariant Hochschild $2$-cocycle $\alpha$) determines a unique (up to $2$-coboundary) equivariant Hochschild $2$-cocycle $\al_\mu$ to serve as the right vertical arrow. Given such a pair of vertical arrows, we find that Drinfeld twist by the cocycle $\mu$ serves as the top horizontal arrow and makes the diagram commute (up to isomorphism).\bb

%particular $1-1$ correspondence between the left- and right-hand verticle edges of the diagram (i.e. between the constant and equivariant Hochschild $2$-cocycles of $S_q(V)\rtimes G$ and $S_{q'}(V')\rtimes G'$ respectively). Claim 3 tells us that given the bottom horizontal edge, and any particular pair of vertical edges given to us through the $1-1$ correspondence in Claim 2, then the same Drinfeld twist serves as the top edge and makes the diagram commute (up to isomorphism).

\nt It should be noted that none of the arrows in this diagram are algebra homomorphisms. This relationship between Drinfeld twists and deformations adds a new perspective to where the twisting result between rational and braided Cherednik algebras in \cite{twistsrcas} comes from. We can see it as arising as a deformation of a more fundamental twisting result between underlying skew group algebras.\bb

\nt This suggests a strategy for finding further quantum Drinfeld Hecke algebras (or braided Cherednik algebras) that are related by a Drinfeld twist - look for skew group algebras related by twist, and deform via suitable Hochschild $2$-cocycles. We apply this strategy in Section \ref{examples_sec} to find a new example of quantum Drinfeld Hecke algebras related by twist.
%that result can be seen as arising, firstly, from the perhaps more fundamental twisting result between skew group algebras as shown in \eqref{twist_skew_sym_algs_1}, and secondly, by this fact that deforming by a Hochschild $2$-cocycle commutes, in the sense of the above diagram, with taking Drinfeld twist.\bb

%fact that their associated graded algebras are related by the same twist, and that we have the correspondence between Hochschild $2$-cocycles given by \eqref{cocycle_map}, which allows for Drinfeld twists between the associated graded algebras to carry over to their deformations.\bb
%rises as the ``lift'' of the twisting result between skew group algebras given in \eqref{twist_of_skew_group_algs} by some Hochschild $2$-cocycle of $S(V\oplus V^*)\rtimes G$.
%$\underline{H}_c(\mu(G))$ can be characterised as a quantum Drinfeld Hecke algebra $H_{-1,\kappa}$ where
%$$\kappa=f_1+\sum_{\epsilon'\in C_{\frac{m}{p}}\backslash \{1\}}(c_{\epsilon'}f_{\epsilon'}+c_1 \tilde{f})$$
%for certain functions $f_{\epsilon'}, \tilde{f}$ arising from \cite{2011arXiv11115243N} (Theorem 7.3). 
%It is important for what follows that the braided Cheredniks arise through Theorem \ref{quantum_dh_alg}, i.e. as the quantum Drinfeld Hecke algebras generated by constant Hochschild $2$-cocycles (assuming $G$ acts on $\bigwedge_q(V)$). In the negative braided case, we think this is true by the following reasoning: \bb
%\nt Recall rational and braided Cherednik algebras are instances of quantum Drinfeld Hecke algebras generated by constant Hochschild $2$-cocycles. 
%This suggests the following:
%\begin{itemize}
%  \item although little is known about how the rep theory of an algebra changes under Drinfeld twist, this task maybe easier to study when the algebras are skew group algebras. Then if anything is known (as I don't know yet) about how the representation theory of an algebra $A$ relates to the representation theory of its deformations over $\Cc [t]$, we might be able to apply the above to learn something about representations of braided Cherednik algebras.
  



\section{Twists of associated graded algebras}\label{claim_1_sec}

In this section we address Claim 1 of the Introduction. First we recall some definitions and properties regarding filtered algebras and their associated graded algebras. A \define{filtered algebra} $Q$ has a filtration $\{F_\bullet\}$, i.e. subspaces $\{0\}=F_{-1}\subset F_0\subset F_1\subset \dots \subset Q$ such that $\cup_{i=0}F_i=Q$ and $F_i \cd F_j\subseteq F_{i+j}$. The \define{associated graded algebra} of $Q$ is given by the space $\gr(Q):=\bigoplus_{i\geq 0}F_i/F_{i-1}$ with product:
\begin{equation}\label{product_ass_gr}
F_i/F_{i-1}\times F_j/F_{j-1}\rt F_{i+j}/F_{i+j-1},\ (x+F_{i-1},y+F_{j-1})\mapsto xy+F_{i+j-1}
\end{equation}

\begin{lemma}\label{isom_filtered} An (iso)morphism of filtered algebras $\phi:Q\rt Q'$ induces an (iso)morphism between their associated graded algebras $\gr(\phi):\gr(Q)\rt gr(Q')$.
\begin{proof}
Suppose the filtered algebras $Q,Q'$ have filtrations $\{F_\bullet\},\{G_\bullet\}$ respectively. A morphism of these algebras is an algebra homomorphism $\phi:Q\rt Q'$ satisfying $\phi(F_i)\subseteq G_i\ \forall i$. Let $\gr(\phi):\gr(Q)\rt \gr(Q')$ be defined in the following way: for each $i\geq 0$ consider the maps $\gr(\phi)^i:F_i/F_{i-1}\rt G_i/G_{i-1},\ v+F_{i-1}\mapsto \phi(v)+G_{i-1}$, and let $\gr(\phi):=\bigoplus_{i\geq 0} \gr(\phi)^i$. Each $\gr(\phi)^i$ is well-defined: if $v+F_{i-1}=v'+F_{i-1}$ then $v-v'\in F_{i-1}$ and $\phi(v-v')=\phi(v)-\phi(v')\in G_{i-1}$, hence $\phi(v)+G_{i-1}=\phi(v')+G_{i-1}$. Clearly $\gr(\phi)$ is a linear map. Finally it is an algebra homomorphism: for $a=x+F_{i-1}, b=y+F_{j-1}$ then applying \eqref{product_ass_gr} we find $\gr(\phi)(ab)=\gr(\phi)(xy+F_{i+j-1})=\phi(xy)+G_{i+j-1}=\phi(x)\phi(y)+G_{i+j-1}=\gr(\phi)(a)\gr(\phi)(b)$, as required.\bb

\nt If $\phi$ is additionally an isomorphism of filtered algebras, i.e. it has an inverse $\phi^{-1}$ that satisfies $\phi^{-1}(G_i)\subseteq F_i$, then we have algebra homomorphisms $\gr(\phi):\gr(Q)\rt \gr(Q')$ and $\gr(\phi^{-1}):\gr(Q')\rt \gr(Q)$. Now $\gr(\phi)\gr(\phi^{-1})(x+G_{i-1})=\gr(\phi)(\phi^{-1}(x)+F_{i-1})=\phi(\phi^{-1}(x))+G_{i-1}=x+G_{i-1}$, and similarly $\gr(\phi^{-1})\gr(\phi)=\id_{\gr(Q)}$. Hence $\gr(\phi^{-1})=\gr(\phi)^{-1}$, and we deduce $\gr(\phi)$ is an algebra isomorphism.
\end{proof}
\end{lemma}

\begin{lemma}\label{ass_is_module_algebra} Suppose $Q$ is a filtered algebra and an $H$-module algebra for a Hopf algebra $H$ acting by degree-preserving endomorphisms. Additionally let $\mu$ be a $2$-cocycle on $H$. Then 
\begin{enumerate}  
  \item the associated graded algebra $\gr(Q)$ has a natural $H$-module algebra structure.
  \item the Drinfeld twist $Q_\mu$ inherits the structure of a filtered algebra from $Q$.
\end{enumerate}
\begin{proof}
(1). Suppose $Q$ has filtration $\{F_\bullet\}$, so $\gr(Q)=\bigoplus_i F_i/F_{i-1}$. Since the action of $H$ is degree-preserving we have $h\rhd x\in F_i\ \forall h\in H,x \in F_i$. Therefore we can define an action of $H$ on $F_i/F_{i-1}$ as $h\rhd x+F_{i-1}=(h\rhd x)+F_{i-1}$. This indeed makes each subspace $F_i/F_{i-1}$ an $H$-module: $(gh)\rhd v+F_{i-1}=((gh) \rhd v)+F_{i-1}=(g\rhd(h\rhd v))+F_{i-1}=g\rhd (h\rhd (v+F_{i-1}))$ and $1\rhd (v+F_{i-1})=(1\rhd v)+F_{i-1}=v+F_{i-1}$. As a direct sum of $H$-modules $\gr(Q)$ is itself an $H$-module.
\iffalse
Let us denote the filtration on $Q$ as $\{F_\bullet\}$, and the action of $H$ on $Q$ as the linear map $\rhd:H\otimes Q\rt Q$. By the degree-preserving assumption we have for each Note $\rhd(h,-):Q\rt Q$ are linear maps for each $h\in H$. These are additionally filtered linear maps by the fact that the action is degree-preserving, i.e. $\rhd(h,F_i)\subseteq F_i$. So by the discussion above, we get induced linear maps $\gr(\rhd(h,-)):\gr(Q)\rt \gr(Q)$ for each $h\in H$, defined as $\gr(\rhd(h,-)):=\bigoplus_{i\geq 0}\gr(\rhd(h,-))^i$ where
$$\gr(\rhd(h,-))^i:F_i/F_{i-1}\rt F_i/F_{i-1}, v+F_{i-1}\mapsto (h\rhd v)+F_{i-1}$$
So we define the action of $H$ on $F_i/F_{i-1}$ as follows, 
$$h\rhd v+F_{i-1}:=\gr(\rhd(h,-))(v+F_{i-1})=(h\rhd v)+F_{i-1}$$
and extend this action linearly to the rest of $\gr(Q)$. %Suppose $v\in Q$ is degree $i$, so $v\in F_i$ but $v\notin F_{i-1}$. Then let $h\rhd v+F_{i-1}:=(h\rhd v)+F_{i-1}$, and by our assumption that the action of $H$ is degree preserving, we know $h\rhd v$ is still of degree $i$. 
We check this firstly makes $\gr(Q)$ an $H$-module.
\begin{itemize}
  \item $gh\rhd (v+F_{i-1})=(gh \rhd v)+F_{i-1}=g\rhd(h\rhd v)+F_{i-1}=g\rhd (h\rhd (v+F_{i-1}))$
  
  \item $1\rhd (v+F_{i-1})=(1\rhd v)+F_{i-1}=v+F_{i-1}$
\end{itemize}
\fi
It is additionally an $H$-module algebra:
\begin{itemize}
  \item The unit of $\gr(Q)$ is given by $1+F_{-1}$ where $1$ is the unit of $Q$ which is degree $0$ in the filtration on $Q$. So: $g\rhd 1+F_{-1}=(g\rhd 1)+F_{-1}=(\epsilon(g)1)+F_{-1}=\epsilon(g)(1+F_{-1})$, since $Q$ is an $H$-module algebra so $h\rhd 1=\epsilon(h)1$ where $\epsilon$ is the counit of $H$.
  
  \item Let $\triangle$ be the coproduct on $H$, and in Sweedler notation $\triangle(h)=h_{(1)}\otimes h_{(2)} \forall h\in H$. Since $Q$ is an $H$-module algebra, $h\rhd vw=(h_{(1)}\rhd v)\cd (h_{(2)}\rhd w)$. Finally,
  \begin{align*}
  h\rhd (v+F_{i-1})\cd (w+F_{j-1}) & = h\rhd (vw+F_{i+j-1})= (h\rhd vw)+F_{i+j-1}\\
    & =(h_{(1)}\rhd v)\cd (h_{(2)}\rhd w)+F_{i+j-1}\\
    & =\bigg((h_{(1)}\rhd v)+F_{i-1} \bigg)\cd \bigg((h_{(2)}\rhd w)+F_{j-1} \bigg)\\
    & = \bigg(h_{(1)}\rhd (v+F_{i-1})\bigg)\cd \bigg(h_{(2)}\rhd (w+F_{j-1})\bigg)
  \end{align*} 
\end{itemize}
\nt Hence all conditions are checked, and $\gr(Q)$ is an $H$-module algebra.\bb

\nt (2). $Q_\mu$ has same underlying vector space as $Q$, so the filtration $\{F_\bullet\}$ on $Q$ is also a filtration on the vector space of $Q_\mu$. If $m$ denotes product on $Q$ and $v\in F_i,w\in F_j$, then we know $m(v\otimes w)\in F_{i+j}$. Since $H$ acts by degree preserving homomorphisms we see $m_\mu(v\otimes w):=m(\mu^{-1}\rhd v\otimes w)\in F_{i+j}$, as required.
\end{proof}
\end{lemma}

%\nt Recall that quantum Drinfeld Hecke algebras are filtered algebras whose associated graded algebras are isomorphic to skew group algebras of the form $S_q(V)\rtimes G$.

\begin{theorem}\label{dh_algebra_to_ass} Let $Q,\underline{Q}$ be quantum Drinfeld Hecke algebras whose associated graded algebras are isomorphic to $S_q(V)\rtimes G$ and $S_{q'}(V')\rtimes G'$ respectively. Suppose further $Q$ is an $H$-module algebra for some Hopf algebra $H$ acting by degree-preserving homomorphisms on $Q$, and that there is a $2$-cocycle $\mu$ of $H$ such that the Drinfeld twist $Q_\mu$ is isomorphic, as a filtered algebra, to $\underline{Q}$. (with filtration on $Q_\mu$ given by Lemma \ref{ass_is_module_algebra} (2)). Then the algebras $S_q(V)\rtimes G$ and $S_{q'}(V')\rtimes G'$ are related by the same twist, i.e.
\begin{equation}\label{dh_algebra_to_ass_eqn}(S_q(V)\rtimes G)_\mu\cong S_{q'}(V')\rtimes G'\end{equation}
\begin{proof}
By Lemma \ref{ass_is_module_algebra}(1) $\gr(Q)$ is an $H$-module algebra, and so $S_q(V)\rtimes G$ also is. Therefore we can consider the Drinfeld twist of $S_q(V)\rtimes G$ by the cocycle $\mu$. We prove \eqref{dh_algebra_to_ass_eqn} via the following sequence of isomorphisms:
\begin{equation}\label{chain_isoms}
  S_{q'}(V')\rtimes G'\cong \gr(\underline{Q})\cong \gr(Q_\mu)\cong \gr(Q)_\mu\cong (S_q(V)\rtimes G)_\mu
\end{equation}
The first and last of these isomorphisms follow by definition of $Q$ and $\underline{Q}$ respectively. The second isomorphism follows by applying Lemma \ref{isom_filtered} with the fact $\underline{Q}\cong Q_\mu$ as filtered algebras (recall $Q_\mu$ was shown to have a filtered structure in Lemma \ref{ass_is_module_algebra}(2)).\bb

\nt Finally we tackle the third isomorphism of \eqref{chain_isoms}. Since $Q_\mu$ and $Q$ share the same underlying vector space and filtration, the algebras $\gr(Q_\mu)$ and $\gr(Q)$ also share the same underlying vector space. This implies $\gr(Q_\mu)$ and $\gr(Q)_\mu$ share the same vector space too. We show the products on $\gr(Q_\mu)$ and $\gr(Q)_\mu$ coincide, deducing these algebras are in fact equal. If $m_\mu$ is the product on $Q_\mu$, the product on $\gr(Q_\mu)$ is given, via \eqref{product_ass_gr}, by
\begin{equation*}\label{mu_gr_product}
\gr(m_\mu)(v+F_{i-1}\otimes w+F_{j-1})=m_\mu(v\otimes w)+F_{i+j-1}
\end{equation*}
If the product on $\gr(Q)$ is given similarly by $\gr(m)(v+F_{i-1}\otimes w+F_{j-1})=m(v\otimes w)+F_{i+j-1}$, we see the product for $\gr(Q)_\mu$ is:
\begin{align*}
\gr(m)_\mu(v+F_{i-1}\otimes w+F_{j-1}) & :=\gr(m)(\mu^{-1}\rhd v+F_{i-1}\otimes w+F_{j-1})\\
& = \gr(m)\bigg(\sum_k \mu_k\rhd (v+F_{i-1})\otimes \mu'_k\rhd (w+F_{j-1}) \bigg)\\
& = \gr(m)\bigg(\sum_k (\mu_k\rhd v)+F_{i-1}\otimes (\mu'_k\rhd w)+F_{j-1}\bigg)\\
& = \sum_k m((\mu_k\rhd v)\otimes (\mu'_k\rhd w))+F_{i+j-1}\\
& = m(\mu^{-1}\rhd v\otimes w)+F_{i+j-1}\\
& = \gr(m_\mu)(v+F_{i-1}\otimes w+F_{j-1})
\end{align*}
where $\mu^{-1}=\sum_k \mu_k\otimes \mu'_k$ for some $\mu_k,\mu'_k\in H$.
%Firstly by Proposition \ref{ass_is_module_algebra} with $A=Q$, we deduce that $\gr(Q)\cong S_q(V)\rtimes G$ is an $H$-module algebra, and so it makes sense to take the Drinfeld twist of $S_q(V)\rtimes G$ by the cocycle $\mu$. Recall the algebras $Q$ and $\gr(Q)\cong S_q(V)\rtimes G$ have isomorphic underlying vector spaces, and similarly for $\underline{Q}$ and $S_{q'}(V')\rtimes G'$. Additionally, by definition of Drinfeld twist, $Q$ and $Q_\mu$ have equal underlying vector spaces, and by our assumption $Q_\mu\cong \underline{Q}$, we deduce $(S_q(V)\rtimes G)_\mu \cong S_{q'}(V')\rtimes G'$ are isomorphic as vector spaces.\bb

%\nt It remains to show the product on $(S_q(V)\rtimes G)_\mu$ coincides with that on $S_{q'}(V')\rtimes G'$. Let $m,\underline{m},m_\mu$ denote the products on $Q,\underline{Q},Q_\mu$ respectively. Since $Q_\mu\cong \underline{Q}$ we have $m_\mu=\underline{m}$. Now $Q$ is of the form $T(V)\rtimes G/\langle v_iv_j-q_{ij}v_j v_i-\kappa(v_i,v_j)\rangle$ where $v_i$ is given degree $1$ and $g\in G$ degree $0$. The filtration on $\underline{Q}$ is given similarly. Since $Q$ and $\underline{Q}$ have isomorphic underlying vector spaces we deduce \textcolor{red}{(Justify)} they have the same filtrations, i.e. $F_i=\underline{F}_i\ \forall i$. Let $v,w\in \underline{Q}$ be degree $i,j$ respectively, then take the following product in $\gr(\underline{Q})$:
%\begin{align*}
%(v+\underline{F}_{i-1})\cd (w+\underline{F}_{j-1}) & = \underline{m}(v\otimes w)+\underline{F}_{i+j-1}\\
%  & = m_\mu(v\otimes w)+\underline{F}_{i+j-1}\\
%  & = m_\mu(v\otimes w)+ F_{i+j-1}
%\end{align*}
%where it can shown that the final expression is the product of $v+F_{i-1}=v+\underline{F}_{i-1}$ and $w+F_j=w+\underline{F}_{j-1}$ in $\gr(Q)_\mu=(S_q(V)\rtimes G)_\mu$.
\end{proof}
\end{theorem}

\begin{cor} The associated graded algebras of the rational and negative braided Cherednik algebras $H_c(G),\underline{H}_{\underline{c}}(\mu(G))$ in Theorem \ref{our_theorem} are related by a Drinfeld twist:
\begin{equation}\label{twist_ass}(S(V\oplus V^*)\rtimes G)_\mu\cong S_{-1}(V\oplus V^*)\rtimes \mu(G)\end{equation}
\begin{proof}
This result follows immediately via Theorems \ref{our_theorem} and \ref{dh_algebra_to_ass}, however we first must actually check the conditions of Theorem  \ref{dh_algebra_to_ass} are satsified in order for us to apply it. Firstly the algebras $H_c(G),\underline{H}_{\underline{c}}(\mu(G))$ are indeed quantum Drinfeld Hecke algebras with associated graded algebras isomorphic to $S(V\oplus V^*)\rtimes G, S_{-1}(V\oplus V^*)\rtimes \mu(G)$ respectively. By Theorem \ref{our_theorem} we know $H_c(G)$ is a $\Cc  T$-module algebra, but it remains to check the action is degree-preserving with respect to the filtration on $H_c(G)$. It is enough to check the action preserves the degree of the generators of $H_c(G)$, which are the elements of the group $G$ (degree $0$), the basis vectors $x_i$ of $V$ and $y_i$ of $V^*$ (degree $1$). For $t\in T$ we have $t\rhd g:=tgt\in G$, so $\deg(t\rhd g)=0=\deg(g)$. Also, $t\rhd x_j=\pm x_j\in V$ and $t\rhd y_j=\pm y_j\in V^*$, so $\deg(t\rhd x_j)=1=\deg(x_j)$ as required, and similarly for $y_j$. So the action is indeed degree-preserving.\bb

\nt Finally we must check the algebra isomorphism $\phi: \underline{H}_{\underline{c}}(\mu(G))\rt (H_c(G))_\mu$ in the proof of Theorem \ref{our_theorem} is an isomorphism of filtered algebras. $\underline{H}_{\underline{c}}(\mu(G))$ is filtered with elements of $\mu(G)$ in degree $0$ and elements $\underline{x}_i,\underline{y}_i$ in degree $1$. Also $\phi$ maps $\underline{x}_i\mapsto x_i$, $\underline{y}_i\mapsto y_i$ and $\Cc  \mu(G)\mapsto \Cc  G$, so it is degree-preserving on the generators of $\underline{H}_{\underline{c}}(\mu(G))$, and therefore a morphism of filtered algebras. Likewise the inverse $\phi^{-1}$ would be degree-preserving on generators, so $\phi$ is isomorphism of filtered algebras.
\end{proof}
\end{cor}


\section{Equivariant Hochschild \texorpdfstring{$2$}{2}-cocycles}\label{claim_2_sec}

In this section we prove Claim 2 from the Introduction. We begin with some definitions.

\begin{definition}\label{equivariant_cocycle} A \define{Hochschild $2$-cocycle} of an algebra $A$ is given by a linear map $\alpha:A\otimes A\rt A$ such that $a \alpha(b\otimes c)+\alpha(a\otimes bc)=\alpha(ab\otimes c)+\alpha(a\otimes b) c\ \forall a,b,c\in A$. Denoting the product on $A$ by $m:A\otimes A\rt A$, we can reexpress this condition as the following equality of maps $A\otimes A\otimes A\rt A$:
\begin{equation}\label{Hochschild_cocycle_1}
m\circ (\id\otimes \alpha)+\alpha\circ (\id\otimes m)=m\circ (\alpha\otimes \id)+\alpha\circ (m\otimes \id)
\end{equation}
%As an aside, notice this is essentially an associativity condition. Indeed every Hochschild $2$-cocycle $\alpha$ defines an ``infinitesimal deformation", which is an associative product on $A[t]/(t^2)$ given by $m'(a\otimes b):=m(a\otimes b)+\alpha(a\otimes b)t$ (see \cite{alma992981689925201631} Chapter 5). %Specialising this at $t=1$ gives an associative algebra whose product can be identified with $m+\alpha$ (\textcolor{red}{not 100\% about this}).\bb
Additionally, a \define{$2$-coboundary} is a Hochschild $2$-cocycle $\alpha$ such that 
\begin{equation}\label{2_coboundary}\alpha(a\otimes b)=m(a\otimes \beta(b))-\beta(m(a\otimes b))+m(\beta(a)\otimes b)\end{equation}
for some linear map $\beta:A\rt A$.
\end{definition}

\nt When $A$ is also an $H$-module algebra for some Hopf algebra $H$ (i.e. an algebra object in $H\Mod$) we will want to distinguish those Hochschild $2$-cocycles that are compatible with the action of $H$:

\begin{definition}\label{equivariant_defn} An \define{$H$-equivariant Hochschild $2$-cocycle} is a Hochschild $2$-cocycle $\alpha$ which commutes with the action of $H$, i.e. 
  \begin{equation}\label{equivariant_action_1}
    h\rhd \alpha(a\otimes b)=\alpha(\triangle(h)\rhd a\otimes b)\hspace{.5cm}\forall h\in H,a,b\in A
  \end{equation}
  In other words $\alpha$ is an $H$-module homomorphism with respect to the natural $H$-module structure on $A\otimes A$. For a group algebra $H=\Cc  G$, for example, this says $g\rhd \alpha(a\otimes b)=\alpha((g\rhd a)\otimes (g\rhd b))\ \forall g\in G$. %\textcolor{red}{Check if this is equivalent to the definition of Hochschild $2$-cocycle in the category $H\Mod$, in the sense of \cite{ARDIZZONI2007297}, and additionally when $H$ is subalgebra of $A$ to $2$-cocycles of ``relative Hochschild cohomology'' of $A$ wrt $u:H\hookrightarrow A$}. 
\end{definition}
\begin{definition}\label{equivariant_coboundary} An \define{$H$-equivariant Hochschild $2$-coboundary} is a $2$-coboundary $\alpha$ satisfying \eqref{2_coboundary} for some linear map $\beta:A\rt A$ which is also an $H$-module homomorphism, i.e. $h\rhd \beta(a)=\beta(h\rhd a)\ \forall h\in H,a\in A$.
\end{definition}

\nt Notice that every $H$-equivariant $2$-coboundary is also $H$-equivariant as a $2$-cocycle, i.e. in the sense of Definition \ref{equivariant_defn}. However a $2$-coboundary that is $H$-equivariant as a $2$-cocycle, i.e. in the sense of Definition \ref{equivariant_defn}, is not neccessarily an $H$-equivariant $2$-coboundary in the sense of Definition \ref{equivariant_coboundary}.\bb
%\nt Since a $2$-coboundary is also a $2$-cocycle, coboundaries can be viewed as $H$-equivariant in the sense of boin both in the  the requirement in Definition \ref{equivariant_coboundary} is stronger requirement than that of the coboundary simply being $H$-equivariant as a $2$-cocycle, in the 

%Notice that a Hochschild $2$-coboundaries can be $H$-equivariant. Since $2$-coboundaries are themselves $2$-cocycles, they can firstly be $H$-equivariant in the sense of Definition \ref{equivariant_defn}. Whereas Definition \ref{equivariant_coboundary} gives a second, stronger, notion of $H$-equivariance. It is indeed stronger since one can check that a $2$-coboundary that is $H$-equivariant in the sense of Definition \ref{equivariant_coboundary} must also be $H$-equivariant in the sense of Definition \ref{equivariant_defn}. Of these two notions of $H$-equivariance, however, it is the stronger one given in Definition \ref{equivariant_coboundary} that we will require in what follows.\bb

\nt An obvious equivalence relation can be defined on the $H$-equivariant Hochschild $2$-cocycles whereby two cocycles are related to each other if their difference is an $H$-equivariant $2$-coboundary (in the sense of Definition \ref{equivariant_coboundary}). We briefly check this is indeed an equivalence relation:
\begin{itemize}
  \item Reflexivity: $\alpha\sim \alpha$ since the difference $\alpha-\alpha=0$ is the trivial $2$-coboundary given by $\beta=0$, where $\beta$ trivially commutes with the action of $H$, as required.
  \item Symmetry: If $\alpha\sim \alpha'$, then $\alpha-\alpha'$ is a $2$-coboundary given by some $H$-module homomorphism $\beta$. Therefore $\alpha'-\alpha$ is also a $2$-coboundary via the homomorphism  using $-\beta$, which also commutes with $H$.
  \item Transitivity: Suppose $\alpha\sim \alpha'$, where $\alpha-\alpha'$ is a coboundary given by $H$-module homomorphism $\beta$, and $\alpha'\sim \alpha''$ is similarly given by a homomorphism $\gamma$. Then $\alpha-\alpha''$ is a coboundary given by the homomorphism $\beta+\gamma$, so we have $\alpha\sim \alpha''$.
\end{itemize}
Let $\HH^2_\rhd (A)$ denote the vector space of equivalence classes of $H$-equivariant Hochschild $2$-cocycles on $A$.

\begin{theorem}\label{cocycle_map_proof} For $H$-module algebra $A$, suppose $\alpha$ is an $H$-equivariant Hochschild $2$-cocycle of $A$ and $\mu$ is a counital $2$-cocycle of $H$. Let $H_\mu$ and $A_\mu$ denote the Drinfeld twists of $H$ and $A$ by $\mu$ respectively. Then the following is an $H_\mu$-equivariant Hochschild $2$-cocycle of $A_\mu$:
$$\alpha_\mu:A_\mu\otimes A_\mu\rt A_\mu,\ a\otimes b\mapsto \alpha(\mu^{-1}\rhd a\otimes b)$$
Additionally, if $\alpha$ is an $H$-equivariant $2$-coboundary of $A$ then $\alpha_\mu$ is an $H_\mu$-equivariant $2$-coboundary of $A_\mu$. Therefore we have a well-defined map $\HH_\rhd^2(A)\rt \HH_\rhd^2(A_\mu),[\alpha]\mapsto [\alpha_\mu]$ between equivalence classes of equivariant cocycles. This is an isomorphism since Drinfeld twist by $\mu^{-1}$ induces the inverse map.

\begin{proof} Let $m$ denote the product on $A$, and by assumption, $\alpha$ satisfies \eqref{Hochschild_cocycle_1}. Firstly we show $\alpha_\mu$ is a Hochschild $2$-cocycle of $A_\mu$, which is the algebra with product defined as: $m_\mu(a\otimes b):=m(\mu^{-1}\rhd a\otimes b)$. Suppose $\mu^{-1}=\sum \mu_1\otimes \mu_2$ for some $\mu_1,\mu_2\in H$ (for better readability we omit the summation indices). In analogy to \eqref{Hochschild_cocycle_1}, $\alpha_\mu$ is a Hochschild $2$-cocycle of $A_\mu$ if,
\begin{equation}\label{Hochschild_cocycle_2}
m_\mu\circ (\id\otimes \alpha_\mu)+\alpha_\mu\circ (\id\otimes m_\mu)=m_\mu\circ (\alpha_\mu\otimes \id)+\alpha_\mu\circ (m_\mu\otimes \id)
\end{equation}
We check this by showing the results of each side applied to an arbitrary element $a\otimes b\otimes c$ coincide. We start by applying the LHS of \eqref{Hochschild_cocycle_2} to $a\otimes b\otimes c$:
\begin{align*}
(m_\mu\circ (\id\otimes \alpha_\mu)+\alpha_\mu\circ & (\id\otimes m_\mu)) (a\otimes b\otimes c) = m(\mu^{-1}\rhd \alpha(\mu^{-1}\rhd a\otimes b)\otimes c)+\alpha(\mu^{-1}\rhd m(\mu^{-1}\rhd a\otimes b)\otimes c)\\
 & = m(\mu_1\rhd \alpha(\mu^{-1}\rhd a\otimes b)\otimes (\mu_2\rhd c))+\alpha(\mu_1\rhd m(\mu^{-1}\rhd a\otimes b)\otimes (\mu_2\rhd c))\\
 &= m(\alpha(\triangle(\mu_1)\mu^{-1}\rhd a\otimes b)\otimes (\mu_2\rhd c))+\alpha(m(\triangle(\mu_1)\mu^{-1}\rhd a\otimes b)\otimes (\mu_2\rhd c))
\end{align*}
where in the final line we apply the $H$-equivariance of $\alpha$, and the fact $A$ is an $H$-module algebra so $m$ commutes with the action of $H$ in exactly the same way as $\alpha$. This now becomes:
\begin{align}\label{cocycle_proof_eq_1}
 & = (m\circ (\alpha\otimes \id)+\alpha\circ (m\otimes \id))[\triangle(\mu_1)\mu^{-1}\rhd (a\otimes b)\otimes (\mu_2\rhd c)]\nonumber\\
 & = (m\circ (\alpha\otimes \id)+\alpha\circ (m\otimes \id))[(\triangle\otimes \id)(\mu^{-1})(\mu^{-1}\otimes 1)\rhd a\otimes b\otimes c]\nonumber\\
& = (m\circ (\id\otimes \alpha)+\alpha\circ (\id\otimes m))[(\id\otimes \triangle)(\mu^{-1})(1\otimes \mu^{-1})\rhd a\otimes b\otimes c]
\end{align}
where in the final line we apply \eqref{Hochschild_cocycle_1} and use the fact that $\mu$ being a counital $2$-cocycle of $H$ implies: 
$$(\id\otimes \triangle)(\mu^{-1})(1\otimes \mu^{-1})=(\triangle\otimes \id)(\mu^{-1})(\mu^{-1}\otimes 1)$$
At this point one takes \eqref{cocycle_proof_eq_1} and essentially does the opposite of all the previous steps in order get an expression written in terms of $m_\mu$ and $\alpha_\mu$ again. Doing this, we arrive at $(m_\mu\circ (\alpha_\mu\otimes \id)+\alpha_\mu\circ (m_\mu\otimes \id))(a\otimes b\otimes c)$ (i.e. the RHS of \eqref{Hochschild_cocycle_2} applied to $a\otimes b\otimes c$), precisely as required. So $\alpha_\mu$ is indeed a Hochschild $2$-cocycle of $A_\mu$.\bb
%-------------------------------------------------------------------------------
% Proving the wrong bloody condition!
\iffalse
For arbitrary $r,s,u\in A$ we have $r\otimes s\otimes u=(\mu\otimes 1)(\triangle\otimes \id)(\mu)\rhd \sum_i r'_i\otimes s'_i\otimes u'_i$ for some $r'_i,s'_i,u'_i\in A$ (in particular they are defined such that $\sum_i r'_i\otimes s'_i\otimes u'_i=(\triangle\otimes \id)(\mu^{-1})(\mu^{-1}\otimes 1)\rhd r\otimes s\otimes u$). Now
\begin{align*}
\big((m_\mu+\alpha_\mu)\otimes \id\big)(r\otimes s\otimes u) & =\big((m+\alpha)\otimes \id\big)((\mu^{-1}\otimes 1)\rhd r\otimes s\otimes u)\\
 & =\big((m+\alpha)\otimes \id\big)((\mu^{-1}\otimes 1)\rhd (\mu\otimes 1)(\triangle\otimes \id)(\mu)\rhd r'\otimes s'\otimes u')\\
 & =\big((m+\alpha)\otimes \id\big)((\triangle\otimes \id)(\mu)\rhd r'\otimes s'\otimes u')
\end{align*}
Next we apply $(m_\mu+\alpha_\mu)$, giving us the left hand side of \eqref{Hochschild_cocycle_2} applied to $r\otimes s\otimes u$,
\begin{align*}
 & = (m+\alpha)\bigg(\mu^{-1}\rhd \big((m+\alpha)\otimes \id\big)((\triangle\otimes \id)(\mu)\rhd r'\otimes s'\otimes u')\bigg)\\
 & = (m+\alpha)\big((m+\alpha)\otimes \id\big)\bigg((\triangle\otimes \id)(\mu^{-1})\rhd (\triangle\otimes \id)(\mu)\rhd r'\otimes s'\otimes u') \bigg)\\
 & = (m+\alpha)\big((m+\alpha)\otimes \id\big)\big( r'\otimes s'\otimes u' \big)
\end{align*}
Moving the second line applies the following identity: $$\mu^{-1}\rhd \big((m+\alpha)\otimes \id\big)(a\otimes b\otimes c)=\big((m+\alpha)\otimes \id\big)((\triangle\otimes \id)(\mu^{-1})\rhd a\otimes b\otimes c)$$
and this identity is fairly easy to prove once we recall the product $m$ commutes with action of $H$, and by assumption, so does $\alpha$, hence: 
$$h\rhd (m+\alpha)(a\otimes b)=(m+\alpha)(\triangle(h)\rhd a\otimes b)\hspace{.5cm}\forall h\in H,a,b\in A$$

%and it is established that the algebra $A[t]/(t^2)$ with product $m+\alpha$ is an $H$-module algebra, and hence  -Justify.}\bb

\nt So we have shown the LHS of \eqref{Hochschild_cocycle_2} applied to $r\otimes s\otimes u$ is equal to $(m+\alpha)\big((m+\alpha)\otimes \id\big)\big( r'\otimes s'\otimes u' \big)$. Similarly we find the RHS of \eqref{Hochschild_cocycle_2} applied to $r\otimes s\otimes u$ is equal to $(m+\alpha)\big(\id\otimes (m+\alpha)\big)(r'\otimes s'\otimes u')$. Finally applying \eqref{Hochschild_cocycle_1}, we find that the LHS and RHS of \eqref{Hochschild_cocycle_2} are indeed equal, so $\alpha_\mu$ is a Hochschild $2$-cocycle of $A_\mu$.\bb
\fi
%End of the proof of the wrong version of hochschild 2cocycle equation.
%-------------------------------------------------------------------------------

\nt Next we check $\alpha_\mu$ is $H_\mu$-equivariant. The coproduct on $H_\mu$ is $\triangle_\mu(h):=\mu\cd \triangle(h)\cd \mu^{-1}$, so
\begin{align*}
h\rhd \al_\mu (a\otimes b) & =h\rhd \al (\mu^{-1}\rhd a\otimes b)\\
& = \al (\triangle(h)\mu^{-1}\rhd a\otimes b)\\
& = \al (\mu^{-1}\triangle_\mu(h)\rhd a\otimes b)\\
& = \al_\mu (\triangle_\mu(h)\rhd a\otimes b)
\end{align*}
as required.\bb %So $\alpha_\mu$ is an $H_\mu$-equivariant Hochschild $2$-cocycle of $A_\mu$.\bb

\nt For the final part, suppose $\alpha$ is an $H$-equivariant $2$-coboundary, i.e. it satisfies \eqref{2_coboundary} for some $\beta:A\rt A$ which commutes with action of $H$. Then, if $\mu^{-1}=\sum_i \mu_i\otimes \mu'_i$ for some $\mu_i,\mu'_i\in H$,
\begin{align*}
  \al_\mu(a\otimes b) & =\al(\mu^{-1}\rhd a\otimes b)\\
  & = \sum_i \al((\mu_i\rhd a)\otimes (\mu'_i\rhd b))\\
  & = \sum_i m(\mu_i\rhd a\otimes \beta(\mu'_i\rhd b))-\beta\circ m(\mu_i\rhd a\otimes \mu'_i\rhd b)+m(\beta(\mu_i\rhd a)\otimes \mu'_i\rhd b)\\
  & = m_\mu(a\otimes \beta(b))-\beta\circ m_\mu(a\otimes b)+m_\mu(\beta(a)\otimes b)
  \end{align*}
where the final equality applies the fact $\beta$ commutes with action of $H$. So indeed $\al_\mu$ satisfies \eqref{2_coboundary} using the same map $\beta$ as for $\al$, here regarded as a linear map $A_\mu \rt A_\mu$ (recall $A_\mu$ has the same underlying vector space as $A$). Also since $H_\mu$ and $H$ have same underlying vector space and the action of $H_\mu$ on $A_\mu$ is the same as that of $H$ on $A$, $\beta$ also commutes under the action of $H_\mu$. So $\al_\mu$ is an $H_\mu$-equivariant $2$-coboundary as required.  
\end{proof}
\end{theorem}

\nt By this theorem we deduce if algebras $A$ and $B$ are related via a Drinfeld twist then there is an isomorphism between the spaces of equivariant Hochschild $2$-cocycles on the two algebras. Next we check the property of being a \textit{constant} Hochschild $2$-cocycle (see Definition \ref{constant}) is preserved under this isomorphism:

\begin{cor} Let $A=S_q(V)\rtimes G$ be an $H$-module algebra for some Hopf algebra $H$ acting by degree-preserving homomorphisms. Suppose $\mu$ is also a counital $2$-cocycle of $H$, and $\alpha$ is a constant and $H$-equivariant Hochschild $2$-cocycle of $A$. Then $\alpha_\mu$ is a constant Hochschild $2$-cocycle of $A_\mu$.
\begin{proof}
Note $\al_\mu(a\otimes b):=\al(\mu^{-1}\rhd a\otimes b)$, and since $H$ acts by degree-preserving homomorphisms on $A$ and $\alpha$ is a degree $-2$ map with respect to the grading on $A$ (which coincides with the grading on $A_\mu$ by Lemma \ref{ass_is_module_algebra}(2)), $\al_\mu$ must also be a degree $-2$ map, and therefore a constant cocycle.
\end{proof}
\end{cor}


%-----------------------------------------------------------------------------


\section{Lifting twists to quantum Drinfeld Hecke algebras}\label{claim_3_sec}
In this section we tackle Claim 3 from the Introduction. We begin in Section \ref{setting_the_scene} by formulating the claim precisely, before proving it in Section \ref{claim_3}.

\subsection{Setting the scene}\label{setting_the_scene}
Let us start by supposing we have the following data:
\begin{itemize}
    \item $A=S_q(V)\rtimes G$, which is an $H$-module algebra for a Hopf algebra $H$ acting by degree-preserving endomorphisms,
    \item $B=S_{q'}(V')\rtimes G'$,
    \item a counital $2$-cocycle $\mu$ of $H$, 
    \item a constant $H$-equivariant Hochschild $2$-cocycle $\alpha$ of $A$, which, by Theorem \ref{cocycle_map_proof}, defines a constant $H_\mu$-equivariant Hochschild $2$-cocycle $\alpha_\mu$ on $A_\mu$,
\end{itemize}
satisfying the following conditions:
\begin{itemize}
    \item $A$ and $B$ are related by a Drinfeld twist by $\mu$, i.e. $A_\mu \cong B$; where this is additionally an isomorphism of graded algebras,
    \item the action of $G$ on $V$ extends to an action on $\bigwedge_q(V)$ by algebra automorphisms so that, by Theorem \ref{quantum_dh_alg}, the cocycle $\alpha$ defines a quantum Drinfeld Hecke algebra $Q_\alpha$ satisfying $\gr(Q_\alpha)\cong A$,
    \item the action of $G'$ on $V'$ extends to an action on $\bigwedge_{q'}(V')$ by algebra automorphisms, which similarly means $\alpha_\mu$ (interpreted now as a Hochschild $2$-cocycle for $B$ using the fact $A_\mu\cong B$) defines a quantum Drinfeld Hecke algebra $Q_{\alpha_\mu}$ satisfying $\gr(Q_{\alpha_\mu})\cong B$.
\end{itemize}
Claim 3 can now be stated in two parts:
\begin{enumerate}[label=(\alph*)]
  \item $Q_\alpha$ is an $H$-module algebra, and is therefore amenable to twisting by the cocycle $\mu$.
  \item The twist of $Q_\alpha$ by $\mu$ is isomorphic to the algebra $Q_{\alpha_\mu}$.
\end{enumerate}

\nt We prove both parts of the claim in Section \ref{claim_3}. However we make several further assumptions, since the claim is not likely to hold in the current level of generality:
\begin{itemize}
   \item let $H=\Cc  T$, the group algebra of a subgroup $T$ of $G$. This is indeed a Hopf algebra with coproduct $\triangle(t)=t\otimes t\ \forall t\in T\subset \Cc  T$. 
   
   \item the action of $H=\Cc  T$ on $A=S_q(V)\rtimes G$ is induced (in a way made explicit shortly) by the following actions of $H$ on $V$ and $\Cc  G$ respectively: 
   \begin{itemize}
      \item Since $T$ is a subgroup of $G$, the action of $\Cc  G$ on $V$ restricts to give an action of $H=\Cc  T$ on $V$. We assume that the set of subspaces spanned by each basis vector $\Omega=\{V_i:=\Cc  v_i\ |\ i\in [n]\}$ forms a system of imprimitivity with respect to this action of $H$. This means that $t(V_i)\in \Omega\ \forall t\in T, i\in [n]$, or more simply: $t(v_i)=\lambda v_j$ for some $\lambda\in \Cc , j\in [n]$. This action of $H$ on $V$ extends to make the tensor algebra $T(V)$ an $H$-module algebra. Furthermore we assume that $q_{ij}=q_{kl}$ whenever $\exists t\in T$ such that $t(V_i)=V_k$ and $t(V_j)=V_l$. 

      \item $H=\Cc  T$ acts on $\Cc  G$ via the adjoint action: $t\rhd g:=tgt^{-1}\ \forall t\in T,g\in G$. By Majid \cite{alma9916633704401631} (Proposition 2.7) this makes $\Cc  G$ an $H$-module algebra.
   \end{itemize}
\end{itemize}

\nt It is quite non-trivial how the two actions of $H$ on $V$ and $\Cc  G$ respectively induce an action of $H$ on $S_q(V)\rtimes G$, so we explain this in detail next. The first step is to apply the following lemma with $A=T(V)$ to deduce that $T(V)\rtimes G$ is an $H$-module algebra.

\begin{lemma}\label{building_h_mod_algebras} Suppose $A$ is a $\Cc  G$-module algebra, and $A\rtimes G$ is the smash product algebra.  If $T$ is a subgroup of $G$, then $A\rtimes G$ has a $\Cc  T$-module algebra structure. 
% Suppose $A$ is an $H$-module algebra for Hopf algebra $H$, and $A\#H$ is the smash product algebra. If $A$ and $H$ are both $\bar H$-module algebras for another Hopf algebra $\bar H$, then $A\#H$ is a $\bar H$-module algebra.
\begin{proof}
First note that on restricting the action of $\Cc  G$ on $A$, $A$ naturally forms a $\Cc  T$-module algebra. Additionally we can make $\Cc  G$ a $\Cc  T$-module algebra via the adjoint action $t\rhd g=tgt^{-1}$ (see Majid \cite{alma9916633704401631} Proposition 2.7). Since $\Cc  T$ is a Hopf algebra we can tensor product the two $\Cc  T$-modules $A$ and $\Cc  G$ together to get a $\Cc  T$-module structure on $A\otimes \Cc  G$, with $T$ acting diagonally on the tensor components. It remains to check this action is compatible with the algebra structure of $A\rtimes G$. Recall the product on $A\rtimes G$ is given by: $(a\otimes g)\cd (a'\otimes g') = (a\cd (g\rhd a'))\otimes gg'$. Now
\begin{align*}
t\rhd (a\otimes g)\cd (a'\otimes g') & = t\rhd (a\cd (g\rhd a'))\otimes gg'\\
& = t\rhd (a\cd (g\rhd a')) \otimes t\rhd gg'\\
& = [(t\rhd a)\cd (tg\rhd a')]\otimes tgg't^{-1}\\
& = (t\rhd a)\cd [(t\rhd g)\rhd (t\rhd a')]\otimes (t\rhd g)\cd (t\rhd g')\\
& = [(t\rhd a)\otimes (t\rhd g)]\cd [(t\rhd a')\otimes (t\rhd g')]\\
& = (t\rhd (a\otimes g))\cd (t\rhd (a'\otimes g'))
\end{align*}
Finally, $t\rhd 1_{A\rtimes G}=t\rhd 1_A\otimes 1_G=t\rhd 1_A\otimes t\rhd 1_G=1_A\otimes 1_G=1_{A\rtimes G}$.
\end{proof}
\end{lemma}
\nt Via the following Lemma we can use the action of $\Cc  T$ on $T(V)\rtimes G$ to induce an action on $S_q(V)\rtimes G$,
\begin{lemma}\label{quotients_h_mod_algs_result} Let $A$ be an $H$-module algebra. 
\begin{enumerate}
  \item If $I$ is an ideal and $H$-submodule of $A$ then $A/I$ is an $H$-module algebra. 
  \item Suppose $A$ is also a filtered algebra and $I$ is any ideal of $A$. Then $A/I$ is also a filtered algebra. Additionally, if the action of $H$ is degree-preserving with respect to the filtration on $A$, then the induced action on $A/I$ is degree-preserving.
\end{enumerate}
\begin{proof}
\begin{enumerate}
  \item It is easy to check to the axioms, for instance the product rule for module algebras follows by: $h\rhd (a+I)\cd (b+I)=h\rhd (ab+I)=(h\rhd ab)+I=(h\rhd a)\cd (h\rhd b)+I=h\rhd (a+I)\cd h\rhd (b+I)$.%Let us consider $A$ as an algebra object in the category $H$-mod. Since the ideal $I$ is an $H$-submodule of $A$, it can also be regarded as an object in $H$-mod. We can therefore take the quotient object of $A$ by $I$ within $H$-mod, which results in the algebra object $A/I$, another algebra object of $H$-mod, and hence an $H$-module algebra.
  \item  If $A$ has filtration $\{F_\bullet\}$, then $A/I$ can easily be shown to have filtration $\{\frac{F_\bullet +I}{I}\}$. Clearly then if the action on $A$ is degree-preserving, i.e. $h\rhd F_i\subseteq F_i$, then the action on $A/I$ is degree-represerving. Indeed suppose $v\in F_i+I$, then we wish to check $h\rhd v+I\in F_i+I$. We have $h\rhd (v+I)=(h\rhd v)+I$, and $h\rhd v\in F_i+I$ using the fact $I$ is a submodule and the action is degree-preserving on $A$.
\end{enumerate}
\end{proof}
\end{lemma}
\nt By Lemma \ref{quotients_h_mod_algs_result}(1), if the 2-sided ideal $I:=\langle  v_i\otimes v_j-q_{ij}v_j\otimes v_i\rangle$ is an $H$-submodule of $T(V)\rtimes G$, then $S_q(V)\rtimes G=T(V)\rtimes G/I$ will be an $H$-module algebra, as we require. Showing $I$ is an $H$-submodule amounts to showing the action of $H$ on $I$ is closed when acting on the generators of $I$. Recall that as part of our assumptions earlier we had that for each $t\in T$, $t\rhd v_i=\lambda v_k,\ t\rhd v_j=\mu v_l$ for some $\lambda,\mu\in \Cc , k,l\in [n]$. Therefore $t\rhd  v_i\otimes v_j-q_{ij}v_j\otimes v_i=\lambda\mu (v_k\otimes v_l-q_{ij}v_l\otimes v_k)$. This is seen to be just another (rescaled) generator of $I$, under the crucial assumption that $q_{ij}=q_{kl}$. This is the reason for our assumption that $q_{ij}=q_{kl}$ whenever $\exists t\in T$ such that $t(V_i)=V_k$ and $t(V_j)=V_l$. Therefore $S_q(V)\rtimes G$ is an $H=\Cc  T$-module algebra. Also, by construction, the action of $H$ on $T(V)\rtimes \Cc  G$ is degree-preserving, and so by Lemma \ref{quotients_h_mod_algs_result}(2), so is the action on $S_q(V)\rtimes G$.

\subsection{Proving Claim 3}\label{claim_3}

\nt With the scene finally set, we prove the first part of Claim 3:

\begin{proposition}\label{is_an_h_mod_alg} Suppose $H=\Cc  T$ and $A=S_q(V)\rtimes G$ are as above, and $\alpha$ is a constant Hochschild $2$-cocycle of $A$ lifting to quantum Drinfeld Hecke algebra $Q_\alpha$. If $\alpha$ is $H$-equivariant then $Q_\alpha$ is an $H$-module algebra.
\begin{proof}
The quantum Drinfeld Hecke algebra that arises from $\alpha$ is given by
$$Q_\alpha:=T(V)\rtimes G/J,\hspace{.5cm} J:=\langle v_i\otimes v_j-q_{ij}v_j\otimes v_i-\kappa (v_i,v_j)\rangle$$
where $\kappa(v_i,v_j)=\sum_{g\in G}\kappa_g(v_i,v_j)=\alpha(v_i\otimes v_j-q_{ij}v_j\otimes v_i)$ (see \cite{2011arXiv11115243N}, proof of Theorem 4.4). By Lemma \ref{building_h_mod_algebras}, $T(V)\rtimes G$ is an $H$-module algebra, so we just need to check the ideal $J$ is an $H$-submodule of $T(V)\rtimes G$ for the result to follow by Lemma \ref{quotients_h_mod_algs_result}(1). We check the action is closed on the generators of $J$, 
\begin{equation}\label{nasty_submodule_eqn}
t\rhd [v_i\otimes v_j-q_{ij}v_j\otimes v_i-\alpha(v_i\otimes v_j-q_{ij}v_j\otimes v_i)]\in J\ \forall t\in T
\end{equation}
Now we must be careful since there are two copies of the expression $v_i\otimes v_j-q_{ij}v_j\otimes v_i$ appearing above, but they are shorthand for two different things. The first copy represents $v_i\otimes v_j\otimes 1_G-q_{ij}v_j\otimes v_i\otimes 1_G\in V^{\otimes 2}\otimes \Cc  G\subseteq T(V)\rtimes G$, whilst the second copy represents $(v_i\otimes 1_G)\otimes (v_j\otimes 1_G)-q_{ij}(v_j\otimes 1_G)\otimes (v_i\otimes 1_G)\in A\otimes A$, where $A=S_q(V)\rtimes G$ and the $v_i,v_j$ are now elements of $S_q(V)$.\bb

\nt Recall $H=\Cc  T$ acts diagonally on $T(V)\rtimes G$, and via the adjoint action on $\Cc  G$, so 
\begin{align*}
t \rhd (v_i\otimes v_j\otimes 1_G-q_{ij}v_j\otimes v_i\otimes 1_G) & = (t\rhd v_i)\otimes (t\rhd v_j)\otimes t1_Gt^{-1} -q_{ij}(t\rhd v_j)\otimes (t\rhd v_i)\otimes t1_Gt^{-1}\\
& = \lambda\mu (v_k\otimes v_l\otimes 1_G-q_{ij}v_l\otimes v_k\otimes 1_G)\\
& = \lambda\mu (v_k\otimes v_l\otimes 1_G-q_{kl}v_l\otimes v_k\otimes 1_G)
%& = \lambda\mu (v_k\otimes v_l-q_{kl}v_l\otimes v_k)
\end{align*}
using the discussion just before the proposition where we assumed $t\rhd v_i=\lambda v_k$ and $t\rhd v_j=\mu v_l$ for some $\lambda,\mu\in \Cc , k,l\in [n]$ and $q_{ij}=q_{kl}$. %The last line is just the shorthand notation for the preceding line. 
Next we apply the fact $\alpha$ is $H$-equivariant,
\begin{align*}
t\rhd \alpha(v_i\otimes v_j-q_{ij}v_j\otimes v_i) & =\alpha\Big(\triangle(t)\rhd \big[(v_i\otimes 1_G)\otimes (v_j\otimes 1_G)-q_{ij}(v_j\otimes 1_G)\otimes (v_i\otimes 1_G)\big]\Big)\\
& = \alpha\Big([t\rhd (v_i\otimes 1_G)]\otimes [t\rhd (v_j\otimes 1_G)]-q_{ij}[t\rhd (v_j\otimes 1_G)]\otimes [t\rhd (v_i\otimes 1_G)]\Big)\\
%& = \alpha\Big([(t\rhd v_i)\otimes 1_G]\otimes [(t\rhd v_j)\otimes 1_G]-q_{ij}[(t\rhd v_j)\otimes 1_G]\otimes [(t\rhd v_i)\otimes 1_G]\Big)\\
& = \lambda\mu\ \alpha\Big((v_k\otimes 1_G)\otimes (v_l\otimes 1_G)-q_{ij}(v_l\otimes 1_G)\otimes (v_k\otimes 1_G)\Big)
%& = \lambda\mu\ \alpha(v_k\otimes v_l-q_{kl}v_l\otimes v_k)
\end{align*}
Now we see that the expression in \eqref{nasty_submodule_eqn} equates to $\lambda\mu(v_k\otimes v_l-q_{kl}v_l\otimes v_k-\alpha(v_k\otimes v_l-q_{kl}v_l\otimes v_k))$, i.e. another (rescaled) generator of $J$, as required. So the action of $H$ on the generators of $J$ is closed, $J$ is an $H$-submodule, and it follows that $Q_\alpha$ is an $H$-module algebra.
\end{proof}
\end{proposition}


%\subsection{Proving Claim 3(b)}\label{claim_3_b}

\nt Now that we know $Q_\alpha$ is an $H$-module algebra, it makes sense to twist it by the cocycle $\mu$ of $H$. We denote the resulting algebra $(Q_\alpha)_\mu$. The next theorem proves the final part of Claim 3.

\begin{theorem}\label{the_main_result} $(Q_\alpha)_\mu\cong Q_{\alpha_\mu}$.

\begin{proof}
We use the following elementary fact to prove the theorem: if $\phi:C\rt D$ is an algebra isomorphism and $J\unlhd C$ is an ideal, then the quotient map $\hat \phi_J: C/J\rt D/\phi(J), c+J\mapsto \phi(c)+\phi(J)$ is an algebra isomorphism. We will take $C=T(V')\rtimes G'$ and $D=(T(V)\rtimes G)_\mu$, and show for a certain choice of ideal $J$ the resulting quotient map $\hat \phi_J$ provides the desired isomorphism $Q_{\alpha_\mu}\xrt{\sim} (Q_\alpha)_\mu$.\bb

\subsubsection{Quotient algebras.} First we wish to characterise the (twisted) skew group algebras $S_{q'}(V')\rtimes G'$ and $(S_q(V)\rtimes G)_\mu$, and their respective (twisted) deformations $Q_{\alpha_\mu},\ (Q_\alpha)_\mu$, as quotient algebras.\bb

\nt Note there exists, by assumption, an isomorphism $\psi:S_{q'}(V')\rtimes G'\xrt{\sim} (S_q(V)\rtimes G)_\mu$ of graded algebras.  Now the algebras $S_{q'}(V')\rtimes G'$ and $Q_{\alpha_\mu}$ are defined as quotients of $T(V')\rtimes G'$ by the ideals $I'=\langle v'_i\otimes v'_j-q'_{ij}v'_j\otimes v'_i\rangle$ and $J'=\langle v'_i\otimes v'_j-q'_{ij}v'_j\otimes v'_i-\alpha'(v'_i\otimes v'_j-q'_{ij}v'_j\otimes v'_i)\rangle$ respectively. Here $\alpha':=\psi^{-1}\circ \alpha_\mu\circ (\psi\otimes \psi)$ is the Hochschild $2$-cocycle on $S_{q'}(V')\rtimes G'$ induced by $\alpha_\mu$ and $\psi$.\bb

\nt We wish to also characterise $(S_q(V)\rtimes G)_\mu$ and $(Q_\alpha)_\mu$ as quotient algebras, in particular of $(T(V)\rtimes G)_\mu$. Consider the ideal $J=\langle v_i\otimes v_j-q_{ij}v_j\otimes v_i-\alpha(v_i\otimes v_j-q_{ij}v_j\otimes v_i)\rangle$ of $T(V)\rtimes G$. The underlying space of $J$ is also an ideal of the algebra $(T(V)\rtimes G)_\mu$. Indeed, let $m$ and $m_\mu$ denote the products on $T(V)\rtimes G$ and $(T(V)\rtimes G)_\mu$ respectively, then we show that $m_\mu(r\otimes x)\in J$ and $m_\mu(x\otimes r)\in J\ \forall r\in (T(V)\rtimes G)_\mu,\ x\in J$. If $\mu^{-1}=\sum \mu_1\otimes \mu_2$ then 
\begin{equation}\label{final_ideal}
m_\mu(r\otimes x) =m(\mu^{-1}\rhd r\otimes x)=m((\mu_1\rhd r)\otimes (\mu_2\rhd x))
\end{equation}
and, by the proof of Proposition \ref{is_an_h_mod_alg}, $J$ is an $H$-submodule, so $\mu_2\rhd x\in J$. Since $J$ is an ideal of $T(V)\rtimes G$ (i.e. with respect to the product $m$) it follows that \eqref{final_ideal} is in $J$ as required. Similarly one shows $m_\mu(x\otimes r)\in J$,  so that $J$ is a $2$-sided ideal of $(T(V)\rtimes G)_\mu$. Next we show $(Q_\alpha)_\mu:=(T(V)\rtimes G/J)_\mu=(T(V)\rtimes G)_\mu/J$.\bb 

\nt As Drinfeld twists do not change the underlying vector space of an algebra, this equality holds at the level of vector spaces. It remains to check the products coincide. Use $[m]$ and $[m_\mu]$ to denote the products on $Q_\alpha=T(V)\rtimes G/J$ and $(T(V)\rtimes G)_\mu/J$ respectively. So for $[a],[b]\in Q_\alpha$ given by some representatives $a,b\in T(V)\rtimes G$, we have: $[m]([a]\otimes [b]):=[m(a\otimes b)]$ and $[m_\mu]([a]\otimes [b]):=[m_\mu(a\otimes b)]$. The product on $(Q_\alpha)_\mu$ is therefore 
\begin{equation}\label{final_proof_eq_1}
[m]_\mu([a]\otimes [b])=[m](\mu^{-1}\rhd [a]\otimes [b])= [m]((\mu_1\rhd [a])\otimes (\mu_2\rhd [b]))
\end{equation}
and at this point we employ the fact that the action of $H$ on $Q_\alpha$ was induced by an action of $H$ on $T(V)\rtimes G$, so $\mu_1\rhd [a]=[\mu_1\rhd a]$ and $\mu_2\rhd [b]=[\mu_2\rhd b]$. Then \eqref{final_proof_eq_1} becomes 
$$[m]([\mu_1\rhd a]\otimes [\mu_2\rhd b])=[m((\mu_1\rhd a)\otimes (\mu_2\rhd b))]=[m_\mu(a\otimes b)]=[m_\mu]([a]\otimes [b])$$
with the last expression being precisely the product on $(T(V)\rtimes G)_\mu/J$. This finishes the proof that $(Q_\alpha)_\mu=(T(V)\rtimes G)_\mu/J$. One can very similarly show that $I:=\langle v_i\otimes v_j-q_{ij}v_j\otimes v_i\rangle$ is an ideal of $(T(V)\rtimes G)_\mu$, and therefore that $(S_q(V)\rtimes G)_\mu=(T(V)\rtimes G)_\mu/I$.\bb

\subsubsection{The isomorphism \texorpdfstring{$\phi$}{phi}.} In this section we construct the isomorphism $\phi:T(V')\rtimes G'\xrt{\sim} (T(V)\rtimes G)_\mu$ required to apply the elementary fact mentioned at the start of the proof. We begin by characterising both sides of $\phi$ as algebra factorisations, in the sense of Majid \cite{alma9916633704401631} (Definition 21.3):
\begin{definition} An \define{algebra factorisation} is an algebra $X$ and subalgebras $A,B\subseteq X$ such that the product map on $X$ defines a linear isomorphism $A\otimes B\cong X$.
\end{definition}
\nt By \cite{alma9916633704401631} (Proposition 21.4) if $X$ has an algebra factorisation into $A,B$, then there exists a linear map $\tau:B\otimes A\rt A\otimes B$ such that $X\cong A\underline{\otimes}B$ as algebras, where $A\underline{\otimes}B$ has underlying vector space $A\otimes B$, unit $1\otimes 1$, and product $(a\otimes b)\cdot (c\otimes d)=a\tau(b\otimes c)d,\ \forall a,c\in A,b,d\in B$. More explicitly, the product on $A\underline{\otimes}B$ is $m_X=(m_A\otimes m_B)\circ (\text{id}_A\otimes \tau\otimes \text{id}_B)$ where $m_A,m_B$ are the products on $A,B$ respectively (given by restricting the product on $X$). We see from the form of $m_X$ this is a natural generalisation of the usual tensor product algebra structure on $A\otimes B$ in which $\tau$ is just the usual swap map.\bb

\nt In \cite{alma998944944401631} (Theorem 7.2.3) we find an explicit description of $\tau$. If $i:A\hookrightarrow X$, $j:B\hookrightarrow X$ are embeddings of the subalgebras, and $\cdot$ is the product on $X$, then by definition of $X$ being an algebra factorisation $\cdot \circ (i\otimes j)$ is a linear isomorphism. Therefore there is a well-defined linear map $\tau$ such that $\cdot \circ (j\otimes i)= \cdot \circ (i\otimes j)\circ \tau$. We see $\tau$ essentially multiplies elements of $B$ with elements of $A$, and then expresses the result in the canonical form of elements in $A\otimes B$.\bb

\nt It is well-known that semidirect products are the archetypal examples of algebra factorisations. Indeed it is easy to check that $T(V')\rtimes G'$ has an algebra factorisation into $T(V')$ and $\Cc G'$ with induced swap map 
\begin{align}\label{tau'}
  \tau':\Cc G'\otimes T(V') & \rt T(V')\otimes \Cc G',\\ 
  g'\otimes v'_1\otimes \dots \otimes v'_k  & \mapsto (g'\rhd v')\otimes\dots \otimes (g'\rhd v'_k) \otimes g'\nonumber
\end{align}
Additionally $(T(V)\rtimes G)_\mu$ has an algebra factorisation into $T(V)_\mu$ and $\Cc G_\mu$. We can see this by noting firstly that $T(V)_\mu$ and $\Cc G_\mu$ are subalgebras of $(T(V)\rtimes G)_\mu$. Let $i:T(V)_\mu\hookrightarrow (T(V)\rtimes G)_\mu, v\mapsto v\otimes 1_G$ and $j:\Cc G_\mu\hookrightarrow (T(V)\rtimes G)_\mu, g\mapsto 1\otimes g$ be the natural embeddings of each of these subalgebras. Also let $m$ and $m_\mu$ denote the products on $T(V)\rtimes G$ and $(T(V)\rtimes G)_\mu$ respectively. Then $(T(V)\rtimes G)_\mu$ will have an algebra factorisation into $T(V)_\mu$ and $\Cc G_\mu$ if $m_\mu\circ (i\otimes j)$ is a linear isomorphism. Here we note $m_\mu\circ (i\otimes j)=m\circ (\mu^{-1}\rhd )\circ (i\otimes j)=m\circ (i\otimes j)\circ (\mu^{-1}\rhd )$, where $\mu^{-1}\rhd$ is the isomorphism on $(T(V)\otimes \Cc G)^{\otimes 2}$ given by acting by $\mu^{-1}$, and in the second equality we apply the fact that $i$ and $j$ are $\Cc T$-module homomorphisms. It is again clear $T(V)\rtimes G$ has an algebra factorisation into $T(V)$ and $\Cc G$, and this makes $m\circ (i\otimes j)$ a linear isomorphism on noting that $i$ and $j$ also define embeddings $T(V)\hookrightarrow T(V)\rtimes G$ and $\Cc G\hookrightarrow T(V)\rtimes \Cc G$. Therefore $m_\mu\circ (i\otimes j)$ is a composition of two linear isomorphisms and must be a linear isomorphism itself.\bb

\nt Now that we know that $(T(V)\rtimes G)_\mu$ has an algebra factorisation, we wish to find the map $\tau$ such that $(T(V)\rtimes G)_\mu=T(V)_\mu \underline{\otimes} \Cc  G_\mu$. As a linear map $\tau:\Cc G\otimes T(V)\rt T(V)\otimes \Cc G$ first sends $g\otimes v_1\otimes \dots \otimes v_k$ to the element $(1\otimes g)\otimes (v_1\otimes \dots \otimes v_k\otimes 1_G)\in (T(V)\otimes \Cc G)^{\otimes 2}$, and then takes the product in $(T(V)\rtimes G)_\mu$,
\begin{align}\label{tau}
\tau(g\otimes v_1\otimes \dots \otimes v_k) & = m_\mu((1\otimes g)\otimes (v_1\otimes \dots \otimes v_k\otimes 1_G))\nonumber\\
  & = m(\mu^{-1}\rhd (1\otimes g)\otimes (v_1\otimes \dots \otimes v_k\otimes 1_G))\nonumber\\
& = m\big(\sum_i [1\otimes (\mu_{(1)i}\rhd g)]\otimes [(\mu_{(2)i}\rhd v_1)\otimes \dots \otimes (\mu_{(2)i}\rhd v_k)\otimes 1_G]\big)\nonumber\\
& = \sum_i \big((\mu_{(1)i}\rhd g)\rhd [(\mu_{(2)i}\rhd v_1)\otimes \dots \otimes (\mu_{(2)i}\rhd v_k)] \big)\otimes (\mu_{(1)i}\rhd g)\\
& \in T(V)\otimes \Cc G\nonumber
\end{align}
where we let $\mu^{-1}=\sum_i \mu_{(1)i}\otimes \mu_{(2)i}$ for some $\mu_{(1)i},\mu_{(2)i}\in \Cc T$.\bb

\nt Next we wish to understand under what conditions two algebra factorisations are isomorphic. Suppose we have two algebra factorisations, $X$ into $A,B$, and $X'$ into $A',B'$, so that$X'\cong A'\underline{\otimes}B'$ and $X\cong A\underline{\otimes}B$. Further suppose we have algebra homomorphisms $\phi_1:A'\rt A$ and $\phi_2:B'\rt B$. Under what condition does the tensor product map $\phi_1\otimes \phi_2:A'\otimes B'\rt A\otimes B$ define an algebra homomorphism  $A'\underline{\otimes}B' \rt A\underline{\otimes}B$? We claim it is precisely if the following holds:
\begin{equation}\label{algebra_hom_condition}
(\phi_1\otimes \phi_2)\circ \tau'=\tau\circ (\phi_2\otimes \phi_1)
\end{equation}
Recall $\phi_1\otimes \phi_2$ is by definition an algebra homomorphism if $(\phi_1\otimes \phi_2)\circ m_{X'}=m_X\circ [(\phi_1\otimes \phi_2)\otimes (\phi_1\otimes \phi_2)]$. Inserting the expanded expression for $m_{X'}$ into this we find:
\begin{align*}
(\phi_1\otimes \phi_2)\circ (m_{A'}\otimes m_{B'})\circ (\text{id}_{A'}\otimes \tau'\otimes \text{id}_{B'}) 
 & =(\phi_1\circ m_{A'})\otimes (\phi_2\circ m_{B'})\circ (\text{id}_{A'}\otimes \tau'\otimes \text{id}_{B'})\\
 & =[(m_A\circ (\phi_1\otimes \phi_2))\otimes (m_B\circ (\phi_1\otimes \phi_2))]\circ (\text{id}_{A'}\otimes \tau'\otimes \text{id}_{B'})\\
 & =(m_A\otimes m_B)\circ (\phi_1\otimes ((\phi_1\otimes \phi_2)\circ \tau')\otimes \phi_2)
\end{align*}
where in the second line we apply the fact $\phi_1,\phi_2$ are algebra homomorphisms. Using \eqref{algebra_hom_condition},
\begin{align*}
 & =(m_A\otimes m_B)\circ (\phi_1\otimes (\tau\circ (\phi_2\otimes \phi_1))\otimes \phi_2)\\
 & =(m_A\otimes m_B)\circ (\text{id}_A\otimes \tau \otimes \text{id}_B)\circ [(\phi_1\otimes \phi_2)\otimes (\phi_1\otimes \phi_2)]\\
  & =m_X\circ [(\phi_1\otimes \phi_2)\otimes (\phi_1\otimes \phi_2)]
\end{align*}
as required. Note that when $\phi_1,\phi_2$ are additionally both isomorphisms, then the tensor product map $\phi_1\otimes \phi_2$ must be a linear isomorphism. When \eqref{algebra_hom_condition} is also satisfied we deduce $X\cong X'$ as algebras.\bb

\nt Now recall that we wished to construct an isomorphism $\phi:T(V')\rtimes G'\xrt{\sim} (T(V)\rtimes G)_\mu$, and so far we have characterised the algebras in the domain and codomain of this map as algebra factorisations. Applying the above, if we can find algebra isomorphisms $\phi_1:T(V')\rightarrow T(V)_\mu$, $\phi_2:\Cc G'\rightarrow \Cc G_\mu$ such that the condition \eqref{algebra_hom_condition} holds, then we will have the desired isomorphism given by $\phi=\phi_1\otimes \phi_2$.\bb

\nt \textcolor{red}{We will construct $\phi_1,\phi_2$ via the graded isomorphism $\psi:S_{q'}(V')\rtimes G'\xrt{\sim} (S_q(V)\rtimes G)_\mu$.}\bb

\nt With $\phi_1,\phi_2$ in hand, and $\tau,\tau'$ given in \eqref{tau}, \eqref{tau'} respectively, we check that \eqref{algebra_hom_condition} holds. Each side of \eqref{algebra_hom_condition} is a linear map $\Cc  G'\otimes T(V')\rightarrow T(V)\otimes \Cc  G$, and a general basis element of $\Cc  G'\otimes T(V')$ is given by $\lambda=g'\otimes v'_1\otimes \dots v'_k$ for some $g'\in G', v'_i\in V, k\in \N$. %Recall that $\tau',\tau$ behave by multiplying elements ``in the wrong order" and then expressing the result in canonical form as a sum of terms ``in the correct order". 
Applying both sides of \eqref{algebra_hom_condition} to $\lambda$ we find:
\begin{align*}
(\phi_1\otimes\phi_2)\circ \tau'(\lambda) & = \phi_1(g'\rhd v'_1)\otimes \dots \otimes \phi_1(g'\rhd v'_k)\otimes \phi_2(g')\\
\tau\circ (\phi_2\otimes \phi_1)(\lambda) & = \sum_i 
   (\mu_{(1)i}\rhd \phi_2(g'))\rhd \big[(\mu_{(2)i}\rhd \phi_1(v'_1))\otimes \dots \otimes (\mu_{(2)i}\rhd \phi_1(v'_k))\big] \otimes (\mu_{(1)i}\rhd \phi_2(g'))
\end{align*}
Checking these are equal reduces to checking:
\begin{equation}\label{identified_actions}
\phi_1(g'\rhd v')=\phi_2(g')\rhd_\mu \phi_1(v')\hspace*{1cm}\forall g'\in G',v'\in V'
\end{equation}
Now $g'\rhd v'\in V'\subset T(V')$, whilst $\phi_1(g'\rhd v')\in V\subseteq T(V)_\mu$. Additionally $V'$ embeds in $S_{q'}(V')\rtimes G'$, as does $V$ into $(S_q(V)\rtimes G)_\mu$. We can therefore identify the subspaces of $T(V')$ and $S_{q'}(V')\rtimes G'$ corresponding to $V'$, and similarly for the subspaces of $T(V)_\mu$ and $(S_q(V)\rtimes G)_\mu$ corresponding to $V$. Restricted to the (identified) subspaces, we have $\phi_1|_{V'}=\psi|_{V'}:V'\rightarrow V$. A similar argument can be made to find $\phi_2|_{\Cc G'}=\psi|_{\Cc G'}:\Cc G'\rightarrow \Cc G_\mu$. Additionally the action of $G'$ on $V'$ is the same whether viewed inside $T(V)'\rtimes G'$ or $S_{q'}(V')\rtimes G'$, and similarly for the twisted action $\rhd_\mu$ of $G$ on $V$. We can therefore say that \eqref{identified_actions} holds if and only if the following holds: $\psi(g'\rhd v')=\psi(g')\rhd_\mu \psi(v')$. This can be verified by using the fact that $\psi$ is an algebra homomorphism:
\begin{align*}
[(\psi(g')\rhd_\mu \psi(v'))\otimes 1_G]\cdot (1\otimes \psi(g')) & =(\psi(g')\rhd_\mu \psi(v'))\otimes \psi(g')\\
& = \psi(g')\cdot \psi(v')\\
& = \psi((g'\rhd v')\otimes g')\\
& = [\psi(g'\rhd v')\otimes 1_G]\cdot (1\otimes \psi(g'))
\end{align*}
and additionally noting the $\psi$ is an isomorphism, so we can multiply from the right by $\psi(g')^{-1}$ to get the desired result. 

\iffalse
Recall we have an isomorphism of graded algebras $\psi:S_{q'}(V')\rtimes G'\xrt{\sim} (S_q(V)\rtimes G)_\mu$. For each $i\in \N$, $\psi$ restricts to an isomorphism of vector spaces between the degree $i$ part of $S_{q'}(V')\rtimes G'$ and the degree $i$ part of $(S_q(V)\rtimes G)_\mu$. The degree $1$ parts of $S_q(V)\rtimes G$ and $S_{q'}(V')\rtimes G'$ are isomorphic to $V$ and $V'$ respectively, so we deduce $V\cong V'$. Therefore these spaces must be of the same dimension, and the map sending $v_i'\mapsto v_i$ must extend linearly to an isomorphism $V'\xrt{\sim}V$. Also, if we note the degree $0$ part of a graded algebra is not just a subspace, but a subalgebra, we see the restriction of $\psi$ to degree $0$ is an isomorphism of \textit{algebras}. The degree $0$ parts of $(S_{q}(V)\rtimes G)_\mu$ and $S_{q'}(V')\rtimes G'$ are $\Cc  G_\mu$ and $\Cc  G'$ respectively, so this restriction of $\psi$ to degree $0$ provides an algebra isomorphism $\Cc  G'\xrt{\sim} \Cc  G_\mu$.\bb

\nt We now define $\phi:T(V')\rtimes G'\rt (T(V)\rtimes G)_\mu$ as the map which sends $v'_i\mapsto v_i, g'\mapsto \psi(g')\ \forall g'\in G'$. We have only specified the map on generators of $T(V')\rtimes G'$, so we must check it extends to a well-defined algebra homomorphism. This means checking that the relations of $T(V')\rtimes G'$ are preserved under $\phi$. The relations include those given by the product structure of $G'$, and the semidirect product relations: $(1\otimes g')\cdot (v'_i\otimes 1_G)=(g'\rhd v'_i)\otimes 1_G \cdot (1\otimes g')$. For the group relations, note $\phi(g')=\psi(g')\ \forall g'\in G'$, so it follows that the group relations are preserved under $\phi$ since they are by $\psi$.

$\phi$ is algebra homomorphism on the subalgebra $\Cc  G'$ since it coincides with $\psi$ and we know $\psi$ is an algebra homomorphism when restricted to degree $0$. For the subalgebra $T(V)$ The relations between the generators of $T(V')\rtimes G'$ are as follows: 
\begin{enumerate}
  \item\label{enum_1} $g\cd g'=gg'\ \forall g,g'\in G'$, where the first product is taken in $T(V)\rtimes G$, and the second inside $G'$, 
  \item\label{enum_2} $g'v'_i=g'(v'_i)g'\ \forall g'\in G',i\in [n]$ 
\end{enumerate}
Since $\phi$ coincides with $\psi$ on the subspace $\Cc  G'$, and $\psi$ restricted to $\Cc  G'$ is an algebra homomorphism, $\phi$ certainly respects the relations between group elements in (\ref{enum_1}), i.e. $\phi(gg')=m_\mu(\phi(g)\otimes \phi(g'))$ as required. As for (\ref{enum_2}), we need to check $m_\mu(\phi(g')\otimes \phi(v'_i))=m_\mu(\phi(g'(v'_i))\otimes \phi(g'))$. This holds because $\phi$ and $\psi$ essentially agree on both $\Cc  G'$ and $V'$, and since this relation holds for $\psi$, it follows for $\phi$ too \textcolor{red}{(this could do with a better explanation)}.\bb

\nt $\phi$ is an isomorphism since the map sending $v_i\mapsto v'_i, g\in G\mapsto \psi^{-1}(g)$ similarly extends to an algebra homomorphism which is clearly inverse to $\phi$. 
\fi

\subsubsection{The quotient map \texorpdfstring{$\hat \phi_{J'}$}{phi}.} 
Recall that above we showed $Q_{\alpha_\mu}=T(V')\rtimes G'/J'$, whilst $(Q_\alpha)_\mu=(T(V)\rtimes G)_\mu/J$. If $\phi$ satisfies $\phi(J')=J$, then the map $\hat \phi_{J'}$ (induced via the elementary fact above) will be an isomorphism $Q_{\alpha_\mu}\xrt{\sim} (Q_\alpha)_\mu$, as required. So we finish by checking $\phi(J')=J$. Recall $J':=\langle v'_i\otimes v'_j-q'_{ij}v'_j\otimes v'_i-\alpha'(v'_i\otimes v'_j-q'_{ij}v'_j\otimes v'_i)\rangle$, where $\alpha':=\psi^{-1}\circ \alpha_\mu\circ (\psi\otimes \psi)$ is the Hochschild $2$-cocycle on $S_{q'}(V')\rtimes G'$ induced by $\alpha_\mu$ and $\psi$. As we did for $I'$ above, 
\textcolor{blue}{Applying the elementary fact which we stated at the start of the proof to this map $\phi$ with the ideal $I':=\langle v'_i\otimes v'_j-q'_{ij}v'_j\otimes v'_i\rangle$ we retrieve the map $\psi$, i.e. $\hat \phi_{I'}=\psi$. Indeed this holds if $\phi(I')=I$. Using the fact $\phi$ is an algebra morphism,
\begin{equation}\label{big_proof_eq_2}
\phi(v'_i\otimes v'_j-q'_{ij}v'_j\otimes v'_i)=\mu^{-1}\rhd (v_i\otimes v_j-q'_{ij}v_j\otimes v_i)
\end{equation}
Now recalling that $\psi$ is an algebra isomorphism sending $v'_i+I'\mapsto v_i+I$, it follows that 
$$[m_\mu]((v_i+I)\otimes (v_j+I))=q'_{ij}[m_\mu]((v_j+I)\otimes (v_i+I))$$
Hence $m_\mu(v_i\otimes v_j-q'_{ij}v_j\otimes v_i)=m(\mu^{-1}\rhd v_i\otimes v_j-q'_{ij}v_j\otimes v_i)\in I$. Now since the action of $H$ is degree-preserving, $\mu^{-1}\rhd v_i\otimes v_j-q'_{ij}v_j\otimes v_i\in V\otimes V$, and $m$ in this case is just the product on $T(V)$, so 
\begin{equation}\label{big_proof_eq_1}
m(\mu^{-1}\rhd v_i\otimes v_j-q'_{ij}v_j\otimes v_i)=\mu^{-1}\rhd v_i\otimes v_j-q'_{ij}v_j\otimes v_i\in I\cap (V\otimes V)
\end{equation}
A general element of $I$ is of the form $\sum_{k,l}r_{kl}(v_k\otimes v_l-q_{kl}v_l\otimes v_k)s_{kl}$ for $r_{kl},s_{kl}\in T(V)\rtimes G$, so elements in $I\cap (V\otimes V)$ take the form $\sum_{kl}\lambda_{kl}(v_k\otimes v_l-q_{kl}v_l\otimes v_k)$ for $\lambda_{kl}\in \Cc $. Therefore by \eqref{big_proof_eq_1} we find:
\begin{equation}\label{big_proof_eq_3}
\mu^{-1}\rhd v_i\otimes v_j-q'_{ij}v_j\otimes v_i=\sum_{k,l}\lambda_{kl}(v_k\otimes v_l-q_{kl}v_l\otimes v_k)
\end{equation}
for some $\lambda_{kl}\in \Cc $. By \eqref{big_proof_eq_2} we then see $\phi(v'_i\otimes v'_j-q'_{ij}v'_j\otimes v'_i)\in I$, so $\phi$ maps generators of $I'$ to elements of $I$ and we deduce $\phi(I')\subset I$. By a similar argument using $\phi^{-1}$ we get $\phi(I')\supset I$, and so $\phi(I')=I$..?}\bb
we show that $\phi$ maps the generators of $J'$ to $J$. Indeed the first part of each generator of $J'$ is of the same form as the generators in $I'$, and we showed above that $\phi(v'_i\otimes v'_j-q'_{ij}v'_j\otimes v'_i)=\sum_{k,l}\lambda_{kl}(v_k\otimes v_l-q_{kl}v_l\otimes v_k)$ for some $\lambda_{kl}\in \Cc $. Now note that $\alpha'(v'_i\otimes v'_j-q'_{ij}v'_j\otimes v'_i)\in \Cc  G'$, and $\phi=\psi$ when both are restricted to $\Cc  G'$. Using the definition of $\alpha'$, 
\begin{align*}
\phi(\alpha'(v'_i\otimes v'_j-q'_{ij}v'_j\otimes v'_i))& =\alpha_\mu\circ (\psi\otimes \psi)(v'_i\otimes v'_j-q'_{ij}v'_j\otimes v'_i)\\
& = \alpha_\mu(v_i\otimes v_j-q'_{ij}v_j\otimes v_i)\\
& = \alpha(\mu^{-1}\rhd v_i\otimes v_j-q'_{ij}v_j\otimes v_i)\\
& = \sum_{k,l}\lambda_{kl}\alpha(v_k\otimes v_l-q_{kl}v_l\otimes v_k)
\end{align*}
where moving to the final line applies \eqref{big_proof_eq_3}. We need to be careful though since the expression $v_i\otimes v_j-q'_{ij}v_j\otimes v_i$ appearing inside the $\alpha$ represents something slightly different to the $v_i\otimes v_j-q'_{ij}v_j\otimes v_i$ appearing in \eqref{big_proof_eq_3}. This was discussed in the proof of Proposition \ref{is_an_h_mod_alg} (see just below \eqref{nasty_submodule_eqn}). \textcolor{red}{However we find that \eqref{big_proof_eq_3} also holds for this second meaning of the expression $v_i\otimes v_j-q'_{ij}v_j\otimes v_i$.} Therefore
$$\phi(v'_i\otimes v'_j-q'_{ij}v'_j\otimes v'_i-\alpha'(v'_i\otimes v'_j-q'_{ij}v'_j\otimes v'_i))=\sum_{k,l}\lambda_{kl}(v_k\otimes v_l-q_{kl}v_l\otimes v_k-\alpha(v_k\otimes v_l-q_{kl}v_l\otimes v_k))\in J$$
and so $\phi(J')\subset J$. \textcolor{red}{Similarly ones shows $\phi(J')\supset J$}, and the result follows.
\end{proof}
\end{theorem}

%-------------------------------------------------------------------------------

\section{Examples}\label{examples_sec}
In the following we give several examples showing how twisting results between skew group algebras can lift to quantum Drinfeld Hecke algebras. In the first example we take the twisting result of the preprint \cite{}, and show that it arises from this lifting process. The groups involved in this example include the Coxeter groups of type $B$, but not the Coxeter groups of type $A$. In the second example we use the lifting process to show that a rational Cherednik algebra of the Coxeter group of type $A_3$ is related by via a new twist to a braided Cherednik algebra.
\begin{example} Our motivating example of Claims 2/3 in work will be the pair of algebras $A=S(V\oplus V^*)\rtimes G$ and $B=S_{-1}(V\oplus V^*)\rtimes \mu(G)$. \textcolor{red}{Note it remains to check that $G$ and $\mu(G)$ act by algebra automorphisms on $\Lambda(V\oplus V^*)$ and $\bigwedge_{-1}(V\oplus V^*)$. In fact in \cite{2011arXiv11115243N} Sections 6 and 7 Witherspoon does not actually check this, and surely without this it isn't guaranteed her constant cocycles generate QDHA's...?}
\end{example}

\begin{example}
  
\end{example}





\iffalse
\subsection*{Next Ideas}

\iffalse
This is probably bullshit.
\begin{proposition} Let $Q$ be an arbitrary filtered algebra with filtration $\{F_\bullet\}$ and whose associated graded algebra $\gr(Q)$ is an $H$-module algebra for some Hopf algebra $H$ acting by degree-preserving homomorphisms. Then $Q$ has a natural $H$-module algebra structure.
\begin{proof}
$H$ acts by degree-preserving homomorphisms on $\gr(Q):=\bigoplus_{i\geq 0}F_i/F_{i-1}$, so for each $h\in H, v\in F_i$, $h\rhd v+F_{i-1}\in F_i/F_{i-1}$. Hence $h\rhd v+F_{i-1}=v_h+F_{i-1}$ for some $v_h\in F_i$ which is unique up to summation by an element of $F_{i-1}$, i.e. $v_h$ defines a unique element of $F_i\backslash F_{i-1}$. Fix $v_h$ to be this unique element of $F_i\backslash F_{i-1}$. For $v\in F_i\backslash F_{i-1}$, let $h\unrhd v:=v_h$, and extend this linearly to $Q$ by noting that every element of $Q$ is a linear combination of vectors from the sets $F_i\backslash F_{i-1}$. We check that $\unrhd$ turns $Q$ into an $H$-module: take $v\in F_i\backslash F_{i-1}$, then
\begin{align*}
  v_{gh}+F_{i-1} & =(gh)\rhd v+F_{i-1}\\
  & = g \rhd (h\rhd F_{i-1})\\
  & = g\rhd (v_h+ F_{i-1})\\
  & = (v_h)_g+F_{i-1}
\end{align*}
so, by uniqueness, we find $v_{gh}=(v_h)_g$. Then 
\begin{align*}
g\unrhd (h\unrhd v)=g\unrhd v_h=(v_h)_g=v_{gh}=(gh)\unrhd v
\end{align*}
as required. Next, $1\rhd v+F_{i-1}=v+F_{i-1}$ so $v_1=v$, and we get $1\unrhd v=v$, again as required. Finally we check this makes $Q$ into an $H$-module algebra: Firstly it is immediate from the fact that $h\rhd 1+F_{-1}=\epsilon(h)(1+F_{-1})$ that $h\rhd 1=\epsilon(h)1$, which is what we needed.  
\end{proof}
\end{proposition}
\fi

Categorical approach:
\begin{itemize}
  \item If $H$ is a Hopf algebra with $2$-cocycle $\chi\in H\otimes H$, then we have a monoidal equivalence of categories $(\ )_\chi: \Alg(H\Mod) \to \Alg(H_\chi\Mod)$  which takes an algebra object $(A,m)$ in $H\Mod$ (i.e. $A$ is an $H$-module algebra) to $(A,m_\chi = m(\chi^{-1}\rhd - ))$, where $H_\chi$ is the Drinfeld twist. 
  \item By \cite{ARDIZZONI2007297}, we can define the Hochschild cohomology of algebra objects inside of a general ``abelian monoidal" category. To check: categories of the form $H\Mod$ are abelian monoidal.
  \item If $F':\mathcal{A}\rt \mathcal{B}$ is an additive equivalence of abelian categories, then the Hochschild cohomology of algebras is preserved over $F$, i.e. we have the following isomorphism of groups: $H_\mathcal{A}^*(A,M)\cong H_\mathcal{B}^*(F(A),F(M))$. Check: the functor $F:H\Mod\xrt{\sim} H_\mu\Mod$ is additive. 
  \item In the case $\Cc =H\Mod$ where $H$ is a subalgebra of $A$, so we have algebra embedding $u:H\rt A$, \cite{ARDIZZONI2007297} (Page 2, see cited source [6] for definition of ``relative Hochschild cohomology'') says the Hochschild cohomology of $A$ as an algebra in the category of $H$-bimodules can be viewed as the ``relative Hochschild cohomology'' of $A$ wrt $u$. Interpret above in terms of relative Hochschild cohomology. \cite{2013arXiv13117124S}, \cite{doi10108000927879908826648} may also be useful.
\end{itemize}

%-------------------------------------------------------------------------------

\subsection*{KZ functors for quantum Drinfeld Hecke algebras} 
Does there exist an analogue of category O for quantum Drinfeld Hecke algebras? Yuri says for braided Cherednik algebras the definition should generalise quite straighforwardly. Perhaps in general can be characterised as braided Drinfeld or Heisenberg doubles, and applying Laugwitz's work.\bb

\nt On Hecke algebra side, Yuri's outlined an approach to constructing a Hecke algebra for mystic reflection groups via categorification. This may lead to an analogue of KZ functor for negative braided Cherednik algebras. This approach uses a cocycle twist of coinvariant algebra. Surely other cocycle twists of the coinvariant algebra would generate, via categorification, other ``quantum Hecke algebras''? Hecke algebra of Coxeter group $W$ is a ``quantisation" (does this mean Hochschild deformation?) of the group ring $\Z W$. Once the quantum Hecke algebras are obtained, would they arise as Hochschild deformations of a ``quantum group ring''. This would mimic the situation from previous section, in which Rational Cherednik algebras (more generally Drinfeld Hecke algebras) are (Hochschild) deformations of their associated graded algebras $S(V)\rtimes G$, and the quantum case arises via Hochschild deformations of $S_q(V)\rtimes G$.

\fi



%\printindex

\bibliographystyle{plain}
\bibliography{mymasterbib}



\end{document}