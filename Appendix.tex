\part{Appendix}
This section contains background material to the previous sections. None of the following is original.

\section{Root systems}\label{roots_for_complex_groups_sec}
We recall here some theory of root systems for real and complex reflection groups relevant to Part \ref{part_1}.\bb

\subsection{Classical root systems.} Root systems are sets of vectors in a Euclidean space, i.e. a real vector space $V$ with positive-definite inner product $(\ ,\ )$. Kane \cite{alma9917011264401631} says they are ``a translation into linear algebra of the geometric configuration formed by the reflecting hyperplanes associated with a reflection group".

\begin{definition}\label{root_system} A \define{root system} in a finite-dimensional Euclidean space $V$ is a finite set $\Phi$ of non-zero vectors spanning $E$, called roots, satisfying the following:
\begin{enumerate}
    \item if $\alpha\in \Phi$, then $\lambda\alpha\in \Phi$ iff $\lambda=\pm 1$,
    
    \item each root $\alpha\in \Phi$ defines a reflection $s_\alpha:V\rt V, v\mapsto v-2\frac{(\alpha,v)}{(\alpha,\alpha)}\alpha$, and we require that if $\alpha,\beta\in \Phi$, then $s_\alpha(\beta)\in \Phi$.
    %\item $\forall \alpha,\beta\in \Phi$, $2\frac{(\alpha,\beta)}{(\alpha,\alpha)}\in \Z$ (the ``integrality condition")
\end{enumerate}
\end{definition}

%\nt The integrality condition is optional, root systems satisfying this are ``crystallographic''. 
\nt The group generated by the reflections $s_\alpha$ is a real reflection group; and is a subgroup of the orthogonal group $\OO(V)$. For crystallographic root systems, this group is called the \define{Weyl group}. Now there is a special subset $\Delta\subset \Phi$ of \define{simple roots} such that every element of $\Phi$ can be expressed as $\Z$-linear combinations of the elements in $\Delta$ with all coefficients of the same sign. The sign determines a choice of positive and negative roots. The set $\{s_\alpha|\ \alpha\in \triangle\}$ is a generating set for the reflection group.

\begin{definition} The \define{Cartan matrix} of a set of simple roots $\alpha_i\in \Delta$ is given by $$A_{ij}=2\frac{(\alpha_i,\alpha_j)}{(\alpha_j,\alpha_j)}$$
\end{definition}

\subsection{Unitary reflection groups.}
The rest of this section is dedicated to summarising the ideas of Lehrer and Taylor \cite{alma9930780234401631} (Chapter 1) in which analogues of root systems for complex reflection groups are defined. Let $V$ be a $\Cc$-vector space.

\begin{definition}\begin{itemize}
  \item A \define{hermitian form} on $V$ is a map $(\ ,\ ):V\times V\rt \Cc$ that is linear in the first argument and satisfies $\overline{(v,w)}=(w,v)$.
  
  \item A hermitian form is \define{positive-definite} if $(v,v)\geq 0\ \forall v\in V$, and $(v,v)=0$ iff $v=0$. Positive-definite hermitian forms are called \define{inner products}.
  
  \item For group $G\subseteq \GL(V)$, a \define{$G$-invariant form} satisfies $(gv,gw)=(v,w)\ \forall g\in V,v,w\in V$.
\end{itemize}
\end{definition}

\nt Note every inner product on $V$ can be described as the canonical one with respect to some basis $e_1,\dots,e_n$, i.e. $(u,v)=a_1\overline{b}_1+\dots +a_n \overline{b}_n$ where $u=a_1e_1+\dots +a_n e_n$, $v=b_1e_1+\dots+b_n e_n$. Additionally, for every finite subgroup $G\subseteq \GL(V)$ there exists a $G$-invariant inner product on $V$.\bb

\begin{definition} For $V$ with inner product $(\ ,\ )$,
\begin{itemize}
  \item $x\in \GL(V)$ is \define{unitary} if $(\ ,\ )$ is $\langle x\rangle$-invariant, for $\langle x\rangle$ the cyclic group generated by $x$.
  
  \item The \define{unitary group} $\UU(V)$ is the subgroup of unitary elements in $\GL(V)$. Every finite $G\subset \GL(V)$ is a subgroup of $\UU(V)$ for some inner product on $V$.
  
  \item $g\in \GL(V)$ is a \define{reflection} if $g$ is of finite order and $\dim(\im(1-g))=1$. If $a\in \im(1-g)$ then for all $v\in V$, $v-gv=\psi(v)a$ for some $\psi(v)\in \Cc$, and this defines a dual vector $\psi\in V^*$ where $\ker(\psi)=\{v\in V|\ gv=v\}$ is the reflecting hyperplane of $g$. 
\end{itemize}
\end{definition}

\nt A reflection $g$ of finite order generates a finite cyclic group $\langle g \rangle$, so by above, $g$ is unitary with respect to some inner product on $V$. If $H$ denotes the reflecting hyperplane of $g$, then wrt a basis adapted to the decomposition $V=\im(1-g)\oplus H$, the matrix representing $g$ takes the form $\text{diag}(\zeta,1,\dots,1)$ for $\zeta$ a primitive $m$-th root of unity.

\begin{definition}\begin{itemize}
  \item A \define{root} of a unitary reflection $g$ is any non-zero vector in the line $\im(1-g)$. A root is \define{short, long, tall} if $(a,a)$ is $1,2$ or $3$ respectively. 
  
  \item A \define{unitary reflection group} is a finite subgroup $\UU(V)$ generated by reflections. 
  
  \item A \define{complex reflection group} is a finite subgroup of $\GL(V)$ generated by reflections. By above, every complex reflection group is a unitary reflection group with respect to some inner product on $V$.
  
  \item For $a\in V\backslash\{0\}$ and $\alpha$ an $m$-th root of unity, we have a corresponding unitary reflection given by 
  $$r_{a,\alpha}(v):=v-(1-\alpha)\frac{(v,a)}{(a,a)}a$$
\end{itemize}
\end{definition}

\nt See Propositions 1.19, 1.20, 1.21, 1.22, 1.23 of \cite{alma9930780234401631} for several properties of unitary reflections.

\begin{definition}\label{definable} A representation $\rho:G\rt \GL(V)$ is \define{definable over a subring} $F$ of $\Cc$ if there exists a basis of $V$ with respect to which the entries of the matrices $\{\rho(g)|g\in G\}$ all belong to $F$.
\end{definition}

\nt For unitary reflection group $G\subset U(V)$, let $r_1,\dots,r_l$ be reflections with corresponding roots $a_1,\dots,a_l$. By above, there exist $\psi_i\in V^*$ s.t. $r_i(v)=v-\psi(v)a_i\ \forall v\in V$.

\begin{definition}
\begin{itemize}
  \item For each pair of roots $a_i,a_j$ as above, the \define{Cartan coefficient} is defined as $\langle a_i |\ a_j \rangle:=\psi_j(a_i)$.\bb
  
  Note: if $r_i$ of order $m$, then $r_i(a_i)=(1-\langle a_i|\ a_i\rangle)a_i=\zeta_m a_i$ where $\zeta_m$ is a primitive $m$-th root of $1$, so $1-\langle a_i|\ a_i\rangle=\zeta_m$.

  \item The \define{Cartan matrix} of reflections $r_1,\dots,r_l$ wrt roots $a_1,\dots,a_l$ is the $l\times l$-matrix $C:=(\langle a_i|\ a_j\rangle)$. This coincides with the usual Cartan matrix when $G$ is a Euclidean reflection group and $a_i$ are the simple roots.
  
  \item The \define{Weyl group} $W(C)$ corresponding to Cartan matrix $C$ is the group generated by the reflections $r_1,\dots,r_l$.
\end{itemize}
\end{definition}

\nt If reflections $r_i$ are unitary, then the roots $a_i$ are orthogonal (wrt to the inner product) to the reflecting hyperplane $\ker(\psi_i)$. If $r_i$ is of order $m_i$, and $\alpha_i$ is a primitive $m_i$-th root of unity, then $r_i=r_{a_i,\alpha_i}$. Additionally, the Cartan coefficients are given by
$$\langle a_i|\ a_j\rangle =(1-\alpha_j)\frac{(a_i,a_j)}{(a_j,a_j)}$$

\nt Now consider a unitary reflection group $G\subset U(V)$ generated by reflections $r_1,\dots,r_l$, with Cartan matrix $C$ corresponding to a choice of roots. If the entries of $C$ lie in a subfield $F$ of $\Cc$, then $G$ is definable over $F$, in the sense of Definition \ref{definable}, using the fact $G$ comes with a representation. For each $G$ there turns out to be a unique minimal choice for such an $F$.

\begin{definition}\begin{itemize}
  \item For group $G\subseteq \GL(V)$, the \define{character} of $G$ is the map $\chi:G\rt \Cc, g\mapsto \tr(g)$.
  
  \item For unitary reflection group $G$, the \define{field of definition} $\Q(G)$ is the field generated by the set $\{\chi(g)|\ g\in G\}$ over $\Q$, i.e. the intersection of all subfields of $\Cc$ that contain both $\{\chi(g)\}$ and $\Q$.
  
  \item An \define{algebraic number field} is a finite extension of $\Q$. Finite means that for such an extension $L$, the degree (i.e. the dimension of $L$ as a $\Q$-vector space) is finite.
  
  \item The \define{ring of integers} of an algebraic number field $K$ is the ring of elements of $K$ that are roots of some monic polynomial with integer(i.e. $\Z\subset \Q$) coefficients.   
  
  \item The \define{ring of definition} $\Z(G)$ is the ring of integers in $\Q(G)$.
  
  \item The \define{cyclotomic fields} are a family of algebraic number fields given by $\Q(\zeta_m)$ where $\zeta_m$ is a primitive $m$-th root of unity. The degree of the extension is $[\Q(\zeta_m):\Q]=\phi(m)$, Euler's phi function. The Galois group $\text{Gal}(\Q(\zeta_m)/\Q)\cong (\Z/m\Z)^\times$, the multiplicative subgroup of the ring of integers mod $m$. The ring of integers is $\Z[\zeta_m]$ (which I assume is the ring of integral polynomials in the ``variable''  $\zeta_m$).
\end{itemize}
\end{definition}

\nt Some results:
\begin{itemize}
  \item The natural representation of every unitary reflection group is definable over its field of definition.
  
  \item For a given generating set of reflections of $G$, there is a Cartan matrix for $G$ with entries in $\Q(G)$. For real reflection groups (Weyl groups) we can pick roots such that the entries of the Cartan matrix are in $\Z$. For unitary reflection groups we ``hope(?)''  that there is a cartan matrix with entries in ring of definition.

  \item A theorem of Brauer states that every complex representation of a finite group is definable over $\Q(\zeta_m)$ where $\zeta_m$ is a primitive $m$-th root of unity such that $g^m=1\ \forall g\in G$.
\end{itemize}



\subsection{\texorpdfstring{$A$}{A}-Root systems.}\label{a_root_sys_sec}

\begin{definition}\begin{itemize}
  \item Let $F$ be a finite abelian extension of $\Q$, i.e. it is an algebraic number field that is also a Galois extension (so has a Galois group), and the Galois group is abelian,
  
  \item let $A$ the ring of integers of $F$,
  
  \item $\mu(A)$ the finite cyclic group of roots of unity $A$,
  
  \item An \define{$A$-root system} in an $F$-vector space $V$ with inner product $(\ ,\ )$, is a pair $(\Sigma,f)$ where $\Sigma$ is finite subset of $V$, and $f:\Sigma\rt \mu(A)$ is s.t.:
  \begin{itemize}
    \item $\Sigma$ spans $V$ and $0\notin \Sigma$,
    \item $\forall a\in \Sigma,\lambda\in F$, have $\lambda a\in \Sigma$ iff $\lambda\in \mu(A)$,
    \item $\forall a\in \Sigma, \lambda\in \mu(A)$, have $f(\lambda a)=f(a)\neq 1$,
    \item $\forall a,b\in \Sigma$ the cartan coefficient $\langle a|\ b\rangle :=(1-f(b))\frac{(a,b)}{(b,b)}$ is in $A$,
    \item $\forall a,b\in \Sigma$, $r_{a,f(a)}(b)\in \Sigma$ and $f(r_{a,f(a)}(b))=f(b)$.
  \end{itemize}
\end{itemize}
\end{definition}

\nt Every irreducible unitary reflection group $G$ has a $\Z(G)$-root system. In particular, consider imprimitive complex reflection group $G=G(m,p,n)$ acting on $\Cc$-vector space $V$ with a $G$-invariant inner product $(\ ,\ )$. Let $\zeta_m$ be primitive $m$-th root of unity, $\mu_m$ be group of $m$-th roots of unity, and $e_1,\dots,e_n$ an orthonormal basis of $V$. Also define,
$$\Sigma(m,m,n):=\{\xi e_i-\eta e_j| \xi,\eta\in \mu_m,\ i\neq j\}$$
and let $f:\Sigma(m,m,n)\rt \mu_2, a\mapsto -1\ \forall a$. 
The reflections of $G(m,m,2)$ are order $2$, are $\Sigma(m,m,n)$ is a $\Z[\zeta_m]$-root system for $G(m,m,n)$ containing long roots.  Additionally, define
$$\triangle:=\{\pm \xi e_i|\ i\in [n],\ \xi\in \mu_m\}$$
with $f:\triangle \rt \mu_m,\ a\mapsto \zeta_m^p\ \forall a$. Finally, when $p\neq m$, let 
$$\Sigma(m,p,n):=\Sigma(m,m,n)\cup \triangle$$

\begin{theorem}[Lehrer, Theorem 2.20] \begin{itemize}
  \item The ring of definition of the dihedral groups $G(m,m,2)$ is $\Z[\zeta_m+\zeta_m^{-1}]\subset \R$, whilst the ring of definition for all other groups $G(m,p,n)$ is $\Z[\zeta_m]$.
  
    \item $(\Sigma(m,p,n),f)$ is a $\Z[\zeta_m]$-root system for $G(m,p,n)$. So by the previous result, except for dihedral groups, this root system is over the field of definition.
\end{itemize}
\end{theorem}









\section{Clifford algebras.} Clifford algebras can be seen as quantizations of exterior algebras, and when generated by real vector spaces, they can also be called geometric algebras. Let $V$ be a finite-dimensional real vector space with a quadratic form $Q:V\rt \R$.

\begin{definition}  The \define{Clifford algebra} $\CL(V,Q)$ (or just $\CL(V)$ when $Q$ is understood) generated by $(V,Q)$ is the quotient of the tensor algebra $T(V)$ by the 2-sided ideal generated by elements of the form $v\otimes v-Q(v)1\ \forall v\in V$. The product on this algebra, often called the geometric product, has the following  form on vectors $x,y\in V\subset \CL(V)$: $xy=(x,y)+x\wedge y$, where $(x,y):=\frac{1}{2}(Q(x+y)-Q(x)-Q(y))$ is the associated symmetric bilinear form. Note $(\ ,\ )$ is positive-definite, and therefore an inner product, whenever $Q$ is positive-definite. Using the anticommutativity of $\wedge$ we notice that $(x,y)=\frac{1}{2}(xy+yx)$.
\end{definition}

\nt Next we define several useful maps on Clifford algebras. Firstly, the linear map on $V$ sending $v\mapsto -v$, preserves the quadratic form $Q$, and so by the universal property of Clifford algebras, extends to an algebra automorphism $\alpha: \CL(V)\rt \CL(V)$. $\alpha$ is an involution, and $\CL(V)$ decomposes into eigenspaces of even and odd parts, $\CL(V)=\CL^{(0)}(V)\oplus \CL^{(1)}(V)$, where
$$\CL^{(i)}(V) :=\{x\in \CL(V)|\ \alpha(x)=(-1)^i x\}$$
This gives $\CL(V)$ its $\Z_2$-grading. Notice also that the tensor algebra $T(V)$ has an antiautomorphism that reverses order of tensor products: $v_1\otimes v_2\otimes \dots \otimes v_k \mapsto v_k\otimes\dots \otimes v_2\otimes v_1$. Since the defining ideal of $\CL(V)$ is invariant under this map, this descends to an antiautomorphism on $\CL(V)$, denoted as transpose $x^t$.

\begin{definition}\label{pin_group_defn}\begin{itemize}
  \item \define{Clifford conjugation} is defined to be the composition of the two operations we have just defined, for $x\in \CL(V)$, $\bar{x}:=\alpha(x^t)=\alpha(x)^t$.

  \item A \define{versor} is an element of $\CL(V)$ of the form $A=\alpha_1\dots \alpha_k$ where $\alpha_i$ are unit vectors. Note that versors are invertible, where the inverse of versor $A$ is given by its Clifford conjugate $\bar{A}=(-1)^k \alpha_k\dots \alpha_1$.
  
  \item The \define{Pin group} $\pin$ is the multiplicative group of versors inside of $\CL(V,Q)$. We also use the notation $\pin_+$, with $\pin_-$ then being used to denote the Pin group of $\CL(V,-Q)$.
\end{itemize}
\end{definition}

\nt The Pin group is a double cover of the orthogonal group $\OO(V)$; in particular, each versor maps to an orthogonal transformation of $V$. Indeed we associate the simplest versors, unit vectors $\alpha\in V$, to the reflections $s_\alpha:V\rt V$ given in Definition \ref{root_system}. Making use of the geometric product on $\CL(V)$, we can reformulate the expression for $s_\alpha$ as follows:
\begin{align*}
  s_\alpha(v) & =v-2(v,\alpha)\alpha\\
  & = v-2\cd \frac{1}{2}(v\alpha+\alpha v)\alpha\\
  & = v-v\alpha^2 -\alpha v\alpha\\
  & = v-v\cd (\alpha,\alpha)1-\alpha v\alpha\\
  & =-\alpha v\alpha
\end{align*}
As pointed out in Dechant \cite{dechant}, the final expression gives a very tidy formula for reflections. Taking several reflections then has the following form: $s_\beta s_\alpha(v)=\beta\alpha v \alpha\beta=\overline{\alpha\beta}v\alpha\beta$. Given a general versor $A=\alpha_1\dots \alpha_k$, the tranformation $\hat{A}(v):=\bar{A}vA$ corresponds to the composition of reflections $s_{\alpha_k} \dots  s_{\alpha_1}$. By the Cartan-Dieudonne theorem, which states that every orthogonal transformation can be described as a composition of reflections, we see that in fact this assignment of versors $A$ to transformations $\hat{A}$ covers the orthogonal group. This is in fact a double cover since versors $A$ and $-A$ generate the same transformation $\hat{A}\in O(V)$.\bb

\nt Even versors, i.e. those in $\CL^{(0)}(V)$, are also called \define{spinors}. The spinors form a subgroup of the Pin group, called the \define{Spin group}! This is a double cover of the special orthogonal group $\SO(V)$.\bb

\nt We will restrict to unitary root systems, i.e. those made up of unit vectors. This way we can view our root systems also being made up of versors, and therefore as subsets of the Pin group. This sets us up for later generalizations. The subgroup of the Pin group generated by a root system is a double cover of the Weyl group of that root system. Note that restricting to unitary root systems is not a great restriction since all real reflection groups have a unitary root system.




\section{Coxeter theory}
We survey some of the basic theory of Coxeter groups and Hecke algebras, which particular focus on the type $B$ case. We are interested in what analogous theory/combinatorics may arise from the mystic presentation of the Weyl group of type $B$.

\begin{definition}
\begin{itemize}
    \item Let $S$ be a set, then $m:S\times S\rt \{1,2,\dots,\infty\}$ is a \define{Coxeter matrix} if $m(s,s')=m(s',s)$ and $m(s,s')=1\iff s=s'$. The same data can equivalently be depicted as a Coxeter graph/diagram.
    
    \item For Coxeter matrix $m$, with $m_{rs}=m(r,s)$, the corresponding \define{Coxeter system} is given by $(W,S)$, where 
    $$W\cong \langle s\in S|\ (sr)^{m_{rs}}=1\ \forall s,r\in S \text{ with } m_{sr}<\infty\rangle$$
      We deduce from the relations that
    \begin{itemize}
      \item $s^2=1\ \forall s\in S$, ``quadratic relation''
    
      \item $srs\dots = rsr\dots$ ($m_{sr}$-terms on each side), $s\neq r$  ``braid relation''
  \end{itemize}
  Any group $G$ which has a presentation of the above form is called a \define{Coxeter group}, and when $S$ is a generating set for $G$ that gives a presentation of the above form we call $(G,S)$ a Coxeter system. Additional reading: see \hyperlink{https://qchu.wordpress.com/2010/06/26/coxeter-groups/}{[Qiaochu1]} and Bjorner's ``Combinatorics of Coxeter groups". 
  \item The \define{rank of Coxeter system} $(W,S)$ is the cardinality $|S|$.
  
  \item The \define{reflections} in Coxeter system $(W,S)$ is the set $T:=\{wsw^{-1}|\ s\in S,w\in W\}=\bigcup_{s\in S}\cc(s)$.\bb 
    Notice how these are all order $2$. But also notice that not all order $2$ symmetries are reflections, for instance rotation by $\pi$. 
  
  \item \define{Weyl groups} are given by the Coxeter groups with $m_{sr}\in \{2,3,4,6\}\ \forall s\neq r$.
  
  \item A Coxeter/Weyl group is \define{simply-laced} when $m_{sr}\in \{2,3\}\ \forall s\neq r$.
    
  \item The Coxeter group of type $B_2$ is 
  $$\langle s,r|\ s^2=r^2=(sr)^4=1, \rangle$$
    This corresponds to Coxeter matrix $\begin{pmatrix}1 & 4 \\ 4 & 1 \end{pmatrix}$. 
    So the set of reflections in this case is $T=\cc(r)\cup \cc(s)=\{r,s,rsr,srs\}$. This is isomorphic to the dihedral group of order $8$. The conjugacy classes of this group are as follows: $\cc(1)=\{1\}$, $\cc(s)=\{s,rsr\}$ (reflections about $y=x$ and $y=-x$ resp), $\cc(r)=\{r,srs\}$ (reflections about $x=0$ and $y=0$ resp), $\cc(sr)=\{sr,rs\}$ (rotation by $\pi/2$ and $3\pi/2$ resp), $\cc(rsrs)=\{rsrs\}$ (rotation by $\pi$).
    We can identify complex reflections with the generators of the above presentation via $s\mapsto s_{12},\ r\mapsto t_1$. This implies $rsr\mapsto s_{12}^{(-1)}, \ srs\mapsto t_2$. Then its complex reflections are as follows:
    $$s_{12}=\begin{pmatrix}0 & 1 \\1 & 0\end{pmatrix},\ 
    s_{12}^{(-1)}=\begin{pmatrix}0 & -1\\-1 & 0\end{pmatrix},\ 
    t_1=\begin{pmatrix}-1 & 0\\ 0 & 1\end{pmatrix},\ 
    t_2=\begin{pmatrix}1 & 0\\ 0 & -1\end{pmatrix}$$
    The root system is:\\
    \begin{figure}[h]
    \includegraphics[scale=.3]{800px-Root_system_B2.png}
    \centering
    \end{figure}
    
    \item For general $n$, the group $B_n$ has Coxeter diagram \tble{B/{}}
    Each node corresponds to a generator, say $s_1,\dots,s_n$ (each of which is order $2$), and the edges indicate: $(s_{n-1}s_n)^4=1$, $(s_i s_{i+1})^3=1\ \forall i<n-1$ and $(s_is_j)^2=1\ \forall |i-j|\geq 2$.
    
    \item $G(m,p,n)=\{wt\in (C_m)^n\rtimes S_n|\ \det(w)\in C_{\frac{m}{p}}\}$ where complex reflections are given by $$\{s_{ij}^{(\epsilon)}| \epsilon\in C_m\}\cup \{t_i^{(\zeta)}|\ \zeta\in C_{\frac{m}{p}}\backslash \{1\}\}$$ with $$s_{ij}^{(\epsilon)}(x_k)=\begin{cases}x_k & k\neq i,j\\ \epsilon x_j & k=i \\ \epsilon^{-1}x_i & k=j\end{cases}\hspace{1cm} t_i^{(\zeta)}(x_k)=\zeta^{\delta_{ik}}x_k$$
    Notice $s_{ij}^{(\epsilon)}$ has order $2$ (regardless of $\epsilon$), and $t_i^{(\zeta)}$ has order $\frac{m}{p}$.

    \item  $\mu(G(2,1,2))$. Notice that $\sigma_{12}^{(1)}=t_1 s_{12}$,  and $\sigma_{12}^{(-1)}=s_{12}t_1$. Using symbols $\sigma,t$, corresponding to $\sigma_{12}^{(1)}$ and $t_1$ respectively, we find this group has the following presentation:
    $$\langle \sigma,t|\ \sigma^4=t^2=(\sigma t)^2=1\rangle$$
    which is clearly not a Coxeter system. Note that $\sigma_{12}^{(1)}$ and $\sigma_{12}^{(-1)}$ lie in the same conjugacy class.
\end{itemize}
\end{definition}

\section{Kazhdan-Lusztig theory}\label{kazhdan_sec}
This section is a survey of Libedinsky \cite{Libedinsky2017GentleIT}.
\begin{definition}
  \begin{itemize}
    \item Each element $x\in W$ has an \definex{expression} as a word in the generators $S$. A \define{reduced expression} is a minimal such word, and the length $l(x)$ is the length of any reduced expression for $x$.
    \item The \define{Bruhat order} is a partial order defined on the elements of a Coxeter system s.t. $x\leq y\iff$there is a subword of a reduced expression for $y$ which is a reduced expression for $x$. Additional reading: \hyperlink{https://qchu.wordpress.com/2010/07/11/chevalley-bruhat-order/}{[Qiaochu2]}.
  \end{itemize}
\end{definition}

\begin{proposition}\label{reduced_braid}\begin{itemize}
  \item Any reduced expression for an element of $W$ can be obtained from any other by applications of the braid relation.
  \item Take $s\in S,w\in W$. If $sx<x$, then $x$ has an expression with $s$ on the far left, i.e. $x=st\dots r$. Similarly $xs<x$ implies there is an expression with $s$ on the right.
\end{itemize}
\end{proposition}

\begin{definition} The \define{Iwahori-Hecke algebra} $H(W)$ associated to finite Coxeter system $(W,S)$ is the $\Z[q,q^{-1}]-$algebra generated by $\{h_s|\ s\in S\}$ subject to relations \begin{itemize}
  \item $h_s^2=(q^{-1}-q)h_s+1$ the ``quadratic relation" 
  
  (we often see instead: $h_s^2=(q-1)h_s+q$, or $(h_s+1)(h_s-q)=0$)
  \item $h_s h_r h_s\dots =h_r h_s h_r \dots $ with $m_{sr}-$terms on each side, the ``braid relation".
\end{itemize}
Setting $q=1$ we retrieve the group algebra $\Z W$. More generally we can define the Hecke algebra of a complex reflection group $W$ as a certain quotient of the braid group $B_W$. This coincides with Iwahori-Hecke algebra when $W$ is a finite Coxeter group. (See Gordon ``SRAs'', Sec 5.15)
\end{definition}

\nt If $x\in W$ has reduced expression $x=st\dots r$, then let $h_x:=h_s h_t\dots h_r$. By Proposition \ref{reduced_braid}, $h_x$ is independent of the reduced expression used. Also let $h_1:=1$.

\begin{lemma}[Iwahori] The set $\{h_x|\ x\in W\}$ is a $\Z[q,q^{-1}]-$basis of $H$, called the \define{standard basis}.
\end{lemma}

\nt The generators $h_s$ for $s\in S$ are invertible, with inverse $h_s^{-1}=h_s+(q-q^{-1})1$. Hence for all $x\in W$ we get $h_x$ is invertible as well.

\begin{definition} The \define{duality} of the Hecke algebra $H$ is the $\Z-$algebra homomorphism $d:H\rt H$ where $q\mapsto q^{-1}, h_x\mapsto (h_{x^{-1}})^{-1}$. We say $x\in H$ is \define{self-dual} if $d(x)=x$.
\end{definition}

\begin{theorem}[Kazhdan-Lusztig] For each $x\in W$ there exists a unique self-dual element $b_x\in H$ s.t. $b_x\in h_x+\sum_{y\in W} q\Z [q]h_y$. Stronger statement holds: take $\sum_{y<x}$ where $<$ is Bruhat order. The set $\{b_x|\ x\in W\}$ is $\Z[q,q^{-1}]-$basis of $H$, called the \define{Kazhdan-Lusztig basis}.\bb

\nt If $b_x=h_x+\sum_{y\in W}h_{y,x}h_y$ for polynomials $h_{y,x}\in \Z[q]$, then the \define{Kazhdan-Lusztig polynomials} are given by $p_{y,x}=q^{l(x)-l(y)}h_{y,x}$.
\end{theorem}

\begin{definition} The \define{braid group} associated to Coxeter system $(W,S)$ is $$B_W:=\langle \sigma_s,\ s\in S|\ (\sigma_s \sigma_r)^{m_{sr}}=1\text{ where }s\neq r\in S, m_{sr}<\infty\rangle$$
\end{definition}


\iffalse
\section{Soergel bimodules}

\subsection{Miemietz on Soergel bimodules}\label{miemietz_sec}
For certain finite Coxeter groups $W$, with reflection representation $V$, there is a Drinfeld twist that can be applied to the coinvariant algebra $S(V)/(S(V)^W)_+$, and the resulting algebra is characterised in terms of the following data:
\begin{itemize}
  \item the group $\mu(W)$, known as the ``mystic reflection group associated to $W$",
  \item the anti-commutative analogue of the symmetric algebra, $S_{-1}(V)$ (this is an example of a quantum (or skew) symmetric algebra).
\end{itemize}
In particular the twisted coinvariant algebra is isomorphic to $S_{-1}(V)/(S_{-1}(V)^{\mu(W)})_+$.\bb

\nt According to Miemietz, we can define the category of Soergel bimodules associated to Coxeter system $(W,S)$ with reflection representation $V$ as follows. Let $R:=\Cc[V]/(\Cc[V]^W)_+$ be the coinvariant ring, and $R_i:=R\otimes_{R^{s_i}}R$ for each simple reflection $s_i\in S$, where $R^{s_i}$ is the invariant subring of $R$ under the action of the reflection $s_i$.

\begin{enumerate}
  \item a) how are coinvariant algebras used to define Soergel bimodules? b) use the ``twisted coinvariant algebras'' in order to define twisted analogues of Soergel bimodules. Show these form a (2-)category.
  \item Does the split Grothendiek group of this category form an algebra? If so, define this as the ``mystic Hecke algebra'' associated to Coxeter group/mystic reflection group. 
  \item Construct corresponding KZ functor, which should map to modules over the mystic Hecke algebra in analogy to the original KZ functor.
  \item Understand how Soergel bimodules are used to construct Khovanov-Rozansky/HOMFLY-PT homology. Construct analogous link homology theory using the twisted Soergel bimodules. Should this be related to the colour Lie algebras?
\end{enumerate}

%------------------------------------------------------------------------------

\subsection{Willamson ``Singular Soergel bimodules''}\label{williamson_sec}
Here we survey \cite{williamson2011singular}. Let $(W,S)$ be a Coxeter system, and $H$ the corresponding Hecke algebra. For $I\subseteq S$ there is a subalgebra $H_I$ of $H$ that is also itself a Hecke algebra. Let $^I H$ be the induced $H$-module from the trivial $H_I$-module.
\begin{definition} The \define{Schur algebroid} is the category with objects given by modules $^I H$ for all \define{finitary} subsets $I\subseteq S$ (i.e. subsets such that the parabolic subgroup $W_I$ is finite \textcolor{red}{(I suppose we aren't assuming $W$ to be a finite Coxeter group, so not given that $W_I$ will be finite in this context)}). Morphisms are morphisms of right $H$-modules.\bb

\nt When $W=S_n$ the Schur algebroid is an ``idempotented version of the q-Schur algebra''. This is where the name Schur algebroid comes from.
\end{definition}

\nt Let \begin{itemize}
  \item $V$ be a \define{reflection faithful} representation of $W$ (i.e. a finite dimensional representation that is faithful and reflections in $W$ are exactly the elements which fix a codimension $1$ subspace of $V$).
  \item $R$ be the graded ring of regular functions on $V$.
  \item for finitary $I\subseteq S$, $R^I:=R^{W_I}$ invariant subring
  \item for finitary $I,J\subseteq S$, $R^I\Mod \text{-} R^J$ the category of graded $R^I\dash R^J$-bimodules.
  \item $\CC$ be the $2$-category with objects  being the finitary subsets $I\subseteq S$, and $\CC(I,J)$ being the category $R^I\Mod \text{-}R^J$.
\end{itemize}

\begin{definition} The \define{category of singular Soergel bimodules} is the full idempotent-complete strict sub-$2$-category of $\CC$ generated by bimodules $R^K\in \ob(R^I\Mod \text{-} R^J)$ for finitary $I\supseteq K\subseteq J$.\bb

\nt More explicitly, for collection of morphisms from $I$ to $J$ in this category is the smallest full additive subcategory of $R^I\Mod \text{-} R^J$ which contains all objects isomorphic to direct summands of shifts\footnote{recall the objects of $R^I\Mod \text{-} R^J$ are graded, so ``shifts" refers to changes to the grading} of objects of the form
$$R^{I_1}\otimes_{R^{J_1}}R^{I_2}\otimes \dots \otimes_{R^{J_{n-1}}}R^{I_n}$$
where $I=I_1\subseteq J_1\supseteq I_2 \subseteq \dots \subseteq J_{n-1} \supseteq I_n=J$ are all finitary.
\end{definition}

\section{Higher categories and Categorification}\label{higher_cats_sec}

\begin{definition} A \define{strict $2$-category} is a category $\CC$ enriched over \textbf{Cat} (the category of categories and functors), meaning that each hom-set is itself a category.\bb

\nt More explicitly, for $A,B\in \ob(\CC)$, the objects and morphisms of the category $\CC(A,B)$ are called $1$-morphisms and $2$-morphisms respectively.
\begin{itemize}
  \item Composition in $\CC(A,B)$ is referred to as \define{vertical composition}. Every $A\in \ob(\CC)$, has an identity morphism $\id_A\in \ob(\CC(A,A))$, and as $\CC(A,A)$ is itself a category, $\id_A$ has its own identity morphism ($2$-morphism) $\id_{\id_A}:\id_A\rt \id_A$. We can identify both $\id_A$ and $\id_{\id_A}$ with $A$. 
  \item For $A,B,C\in \ob(\CC)$ there is a functor $\circ:\CC(B,C)\times \CC(A,B)\rt \CC(A,C)$ called \define{horizontal composition}. This is ``associative", meaning the two ways of horizontally composing $\CC(C,D)\times \CC(B,C)\times \CC(A,B)$ down to $\CC(A,D)$ are equivalent. Note this can be weakened to associativity up to $2$-isomorphism, resulting in a so called \define{bicategory}.
\end{itemize}
Vertical and horizontal composition satisfies the ``interchange law'', which is depicted as:\bb
\textcolor{red}{add image}
\end{definition}

\nt An \define{idempotent} is an endomorphism $e:X\rt X$ such that $e^2=e$. If there is a subobject $i:A\hookrightarrow X$ with a projection $p:X\rt A$ such that $e=i\circ p$ and $\id_A=p\circ i$, then the idempotent $e$ is said to be \define{split}. A category is \define{idempotent complete} if every idempotent splits. For preadditive categories this can be rephrased as an idempotent $e:X\rt X$ splits precisely if it has a kernel. In this case, there exists a direct sum decomposition $X=Y\oplus Z$ such that the projection $p$ projects onto $Z$, which is itself an object in the category. Then being an idempotent complete category says that all direct summands of objects are contained in your category. For more see \hyperlink{https://stacks.math.columbia.edu/tag/09SF}{Stack: Karouian categories} and \hyperlink{https://ncatlab.org/nlab/show/idempotent}{NLAB: Idempotents: 1 Idea.}\bb
  
\nt If a category $\CC$ is not idempotent complete then it can be ``completed" via the following construction:
\begin{definition}\label{karoubi_def} The \define{Karoubi envelope} of $\CC$ is the category $\text{Split}(\CC)$ whose objects are pairs $(A,e)$ for $A\in \ob(\CC)$ and $e:A\rt A$ an idempotent in $\CC$. A morphism from $(A,e)$ to $(A',e')$ is given a morphism $f:A\rt A'$ in $\CC$ such that $e'\circ f=f=f\circ e$. Composition in $\text{Split}(\CC)$ is as in $\CC$, however the identity morphism on $(A,e)$ is given by $e$ (not $\id_A$). 
\end{definition}

\nt $\CC$ embeds in $\text{Split}(\CC)$, and every idempotent in $\text{Split}(\CC)$ splits (see \hyperlink{https://en.wikipedia.org/wiki/Karoubi_envelope}{Wiki} for more.)\bb

\nt Useful info on Categorification found in: Baez's "Categorification", and Khovanov's ``Introduction to categorification" course at Columbia which looks as: 
\begin{itemize}
  \item TQFTs, 
  \item Categorification of knot polynomials, 
  \item Grothendiek groups, and 
  \item biadjoint functors ("Cubes of frob exts" says an extension $A\subseteq B$ of commutative algebras is Frobenius if and only if the functors $\text{Ind}^B_A$ and $\text{Res}^B_A$ are biadjoint)
\end{itemize}

\iffalse
\begin{itemize}
  \item TQFT 
  \begin{itemize}
    \item 1d TQFT's (oriented vs unoriented)
    \item 2d TQFT's (oriented, unoriented, module-theoretic characterisation of Frobenius algebras, thin surface TQFTs as example of non-commutative (but still symmetric) Frobenius algebras)
  \end{itemize}
  
  \item Categorification of the Jones polynomial
  \begin{itemize}
    \item Kauffman bracket
    \item Localisation of Kauffman bracket
    \item Idea behind categorification
    \item Categorification of Jones polynomial
    \item Tate conjecture
  \end{itemize}
  
  \item Grothendiek groups\begin{itemize}
    \item $G_0$
    \item $K_0$
    \item Pairing between $K_0$ and $G_0$.
  \end{itemize}
  
  \item Extending link homology to tangles
  
  \item Milnors Conjecture
  
  \item Biadjoint functors
\end{itemize}
\fi
\fi